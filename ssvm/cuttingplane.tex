%\documentclass[10pt,a4paper]{article}
%\documentclass[twocolumn,10pt,a4paper]{article}
%\documentclass[twocolumn,a4wide,9pt]{extarticle}
\documentclass[9pt]{extarticle}
\usepackage[a4paper,top=0.85in,left=0.75in,bottom=1in,right=0.52in]{geometry}
\usepackage{hyperref}
\usepackage{amsmath}
\usepackage{amsfonts}
\usepackage{bm}
\usepackage{algorithm}
\usepackage{algorithmic}
\usepackage[sc]{mathpazo}
\linespread{1.05}         % Palladio needs more leading (space between lines)
\usepackage[T1]{fontenc}

\DeclareMathOperator*{\argmin}{argmin}
\DeclareMathOperator*{\argmax}{argmax}
\newcommand{\eat}[1]{}
\setlength{\columnsep}{1.5em} % spacing between columns

\title{Notes on Cutting-plane Methods}

\author{Dawei Chen}

\date{\today}

\begin{document}

\maketitle

\section{Problem Setting}
\label{sec:problem}

The goal of cutting-plane methods is to find/localise a point in a convex \textit{target set} $Z \in \mathbb{R}^n$,
or determine that $Z$ is empty in some cases. 
The method does not assume any direct access to the description of $Z$,
such as the objective and constraint functions in an optimisation problem, except through a \textit{cutting-plane oracle}.
The method generates a query point $q$ and pass it to the oracle, 
the oracle either tells us that $q \in Z$ (in which case we are done), or it returns a hyperplane which separates $q$ from $Z$.
This hyperplane is called a \textit{cutting-plane}, or \textit{cut}, since it eliminates a half-space from our search.

Cutting-plane methods are also known as \textit{localisation} methods. 
A conceptual description of cutting-plane methods is shown in Algorithm~\ref{alg:cutting-plane}.


\begin{algorithm}[htbp]
\caption{Cutting-plane algorithm}
\label{alg:cutting-plane}
\begin{algorithmic}[1]
\STATE \textbf{Given}: an initial polyhedron $\mathcal{P}_0$ that contains $Z$.
\STATE $k = 0$
\REPEAT
    \STATE Generate a query point $q^{(k+1)}$ in $\mathcal{P}_k$
    \STATE Query the oracle at $q^{(k+1)}$
    \IF{~The oracle determines that $q^{(k+1)} \in Z$~}
        \RETURN $q^{(k+1)}$
    \ELSIF{~The oracle returns a cutting-plane $a_{k+1}^\top z \le b_{k+1}$~}
        \STATE Update constraints: $\mathcal{P}_{k+1} = \mathcal{P}_k \cap \{z | a_{k+1}^\top z \le b_{k+1} \}$
    \ENDIF
    \STATE $k = k + 1$
\UNTIL{Convergence or $\mathcal{P}_{k+1} = \emptyset$}
\end{algorithmic}
\end{algorithm}


\noindent
For a convex optimisation problem with $m$ constraints,

\begin{equation}
\label{eq:cvxprob}
\begin{aligned}
\min_{x} ~& f_0(x)        & \\
s.t.~~   ~& f_i(x) \le 0, & i = 1, \dots, m
\end{aligned} 
\end{equation}
where $f_0, \dots, f_m$ are convex and differentiable, the target set $Z$ is the optimal (or $\varepsilon$-suboptimal) set.

Given a query point $q$, the oracle first checks for feasibility.
If $q$ is not feasible, this means that at least one constraint in problem (\ref{eq:cvxprob}) is violated.
Suppose constraint $f_j(x) \le 0$ is violated by $q$, then we have $f_j(q) > 0$.
In addition, as $f_j(x)$ is convex and differentiable, we have the inequality
\begin{equation}
\label{eq:funprop}
f_j(x) \ge f_j(q) + \nabla f_j(q) (x - q),~ j = 0, \dots, m
\end{equation}
We conclude that if $f_j(q) + \nabla f_j(q) (x - q) > 0$, then $f_j(x) > 0$, which violated the constraint $f_j(x) \le 0$ in problem (\ref{eq:cvxprob}).
Thus, any feasible point satisfies the inequality
\begin{equation}
\label{eq:feacut}
f_j(q) + \nabla f_j(q)^\top (x - q) \le 0.
\end{equation}
This is called a \textit{feasibility cut} for problem (\ref{eq:cvxprob}) since it cuts away the half-space 
$\{z | f_j(q) + \nabla f_j(q)^\top (z - q) > 0 \}$ with infeasible points.
If more than one constraint is violated by $q$, we can generate a \emph{feasibility cut} for each violated constraint.

On the other hand, if $q$ is feasible, suppose $\nabla f_0(q) \ne 0$ (otherwise $q$ is optimal and we are done),
as $f_0(x)$ is differentiable, we have
\begin{equation*}
f_0(x) > f_0(q), \text{~if~} \nabla f_0(q) > 0 \text{~and~} x > q,
\end{equation*}
similarly,
\begin{equation*}
f_0(x) > f_0(q), \text{~if~} \nabla f_0(q) < 0 \text{~and~} x < q.
\end{equation*}
It means that 
\begin{equation*}
f_0(x) > f_0(q), \text{~if~} \nabla f_0(q)^\top (x - q) > 0.
\end{equation*}
In other words, any point that satisfies inequality $\nabla f_0(q)^\top (x - q) > 0$ has an objective value larger than $f_0(q)$ 
and hence cannot be optimal.
It follows that we can form a cutting-plane
\begin{equation}
\label{eq:objcut}
\nabla f_0(q)^\top (x - q) \le 0,
\end{equation}
which is called an \textit{objective cut} for problem (\ref{eq:cvxprob}) and 
it cuts out the half-space $\{z | \nabla f_0(q)^\top (z - q) > 0 \}$ with non-optimal points.

If we keep track of the best (smallest) objective value $f_\text{best} = f_0(q_\text{best})$ for all feasible query points during the querying, 
%since all other feasible points have objective values at least $f_\text{best}$, 
since the optimal feasible point has an objective value at most $f_\text{best}$, 
we can cut off the half-space of points $\{z | f_0(z) > f_\text{best} \}$ with objective values greater than $f_\text{best}$, 
which means we add a constraint $f_0(z) \le f_\text{best}$, and considering Equation~(\ref{eq:funprop}), 
we have a deep objective cut~\cite{boydlocalization},
\begin{equation}
\label{eq:deepobjcut}
f_0(q) + \nabla f_0(q) (x - q) - f_\text{best} \le 0,
\end{equation}
where $q$ is the current query point. If $q = q_\text{best}$, this cut reduced to the objective cut~(\ref{eq:objcut}).

%If $q$ is feasible and $\nabla f_0(q) = 0$ then $q$ is optimal.
For non-differentiable problems, the gradients $\nabla f_j(z)$ can generally be replaced by sub-gradients.


\section{Generate query points}
\label{sec:query}

We would like to generate a query point $q^{(k+1)}$ in the current polyhedron $\mathcal{P}_{k}$ such that 
the resulting cut reduces the size of $\mathcal{P}_{k+1}$ as much as possible.
However, when we query the oracle at point $q^{(k+1)}$, we do not know in which direction the generated cut will be excluded.
If we measure the informativeness of the $k$-th cut using the volume reduction ratio $\frac{V(\mathcal{P}_{k+1})}{V(\mathcal{P}_{k})}$,
we seek a point $q^{(k+1)}$ such that, no matter which direction to cut (returned by the oracle), we can obtain a certain guaranteed volume reduction.


\subsection{Method of Kelley-Cheney-Goldstein}
\label{sec:kcg}

Given query points $q^{(1)}, \dots, q^{(k)}$, 
one approach to choose the next query point $q^{(k+1)}$ is to greedily use the vertex of the current polyhedron $\mathcal{P}_k$ 
that minimises the objective, which can be found by solving~\cite{wulff2013analytic}
\begin{equation}
\label{eq:kcg}
\begin{aligned}
\min_{z} ~& \theta  \\
s.t.~~   ~& \theta \ge f_0(q^{(i)}) + \nabla f_0(q^{(i)})^\top (z - q^{(i)}),~ \forall i \le k \\
          & a_i^\top z \le b_i,~ \forall i \le k \\
          & f_j(z) \le 0,~ j = 1, \dots, m
\end{aligned}
\end{equation}
where $a_i, b_i$ are the set of existing cutting planes returned by querying the oracle at points $q^{(i)}, i \le k$.


\subsection{Chebyshev center method}
\label{sec:chebyshev}

If we rescale the gradients $\nabla f_0(q^{(i)})$ to unit length in problem (\ref{eq:kcg}), 
it results in finding the center of the largest Euclidean ball that lies inside the current polyhedron $\mathcal{P}_k$~\cite{wulff2013analytic},
in other words, we find the next query point $q^{(k+1)}$ by solving
\begin{equation}
\label{eq:chebyshev}
\begin{aligned}
\min_{z} ~& \theta  \\
s.t.~~   ~& \theta \ge f_0(q^{(i)}) + \frac{\nabla f_0(q^{(i)})}{\|\nabla f_0(q^{(i)})\|} ^\top (z - q^{(i)}),~ \forall i \le k \\
          & a_i^\top z \le b_i,~ \forall i \le k \\
          & f_j(z) \le 0.~ j = 1, \dots, m
\end{aligned}
\end{equation}
This variant is called the Chebyshev center method, which is shown to have significantly better convergence properties than the method of Kelley-Cheney-Goldstein~\cite{goffin2002convex}.


\subsection{Analytic center cutting plane method}
\label{sec:accpm}

Given a linear constraint $a_i^\top z \le b_i$, we define a slack variable $s_i \in \mathbb{R}$ as $s_i = b_i - a_i^\top z$,
that is, $s_i$ measures how far the current solution is from the constraint.
The analytic center is defined as the unique maximiser of the function~\cite{wulff2013analytic}
\begin{equation}
\label{eq:accpm}
\argmax_z \prod_i s_i = \argmax_z ~ \sum_{i=1}^k \log(b_i - a_i^\top z) + \sum_{j=1}^m \log(d_j - c_j^\top z),
\end{equation}
where we assume constraints $f_j(z) \le 0$ in problem (\ref{eq:cvxprob}) are linear and rewrite them as $c_j^\top z \le d_j$.
The unique maximiser of (\ref{eq:accpm}) can be efficiently found using Newton iterations~\cite{goffin2002convex}.

The analytic center cutting plane method (ACCPM) chooses the analytic center of polyhedron 
\begin{equation*}
\mathcal{P}_k = \{ z | c_j^\top z \le d_j, ~ j=1, \dots, m \text{~and~} a_i^\top z \le b_i, ~ i=1, \dots, k \}
\end{equation*}
to query the oracle.
ACCPM seems to give a good trade-off in terms of simplicity and practical performance~\cite{boydlocalization}.


\subsection{Center of gravity/Bayes point method}
\label{sec:cg}

Assume set $\mathcal{C} \subseteq \mathbb{R}^n$ is bounded and has nonempty interior. 
The center of gravity of $\mathcal{C}$ is defined as
\begin{equation}
\textbf{cg}(\mathcal{C}) = \frac{\int_\mathcal{C} z dz}{\int_\mathcal{C} dz}.
\end{equation}

The center of gravity (CG) method chooses the point $q^{(k+1)} = \textbf{cg}(\mathcal{P}_{k})$ to query the oracle~\cite{louche2015cutting}.
It turns out that this method has a very good convergence property in terms of the worst-case volume reduction factor,
in particular, we always have
\begin{equation}
\frac{V(\mathcal{P}_{k+1})}{V(\mathcal{P}_{k})} \le 1 - \frac{1}{e} \approx 0.63,
\end{equation}
in other words, the volume of the localisation polyhedron is reduced by at least $37\%$ at each iteration~,
and this guarantee is completely independent of all problem parameters, including the dimension $n$.
However, it is \textit{extremely difficult} to compute the center of gravity of a polyhedron in $\mathbb{R}^n$, described by a set of linear inequalities,
which makes this method impractical.
Variants that compute an approximate center of gravity have been developed, and some of these approximations can be used to create a practical CG method~\cite{boydlocalization}.


\section{Train structured SVM using cutting-plane methods}
\label{sec:ssvm}


\subsection{Train the $n$-slack formulation of structured SVM}
\label{sec:nslackssvm}


Given $n$ training examples $(\mathbf{x}_1, \mathbf{y}_1), \dots, (\mathbf{x}_n, \mathbf{y}_n)$, 
the structured SVM with margin-rescaling\footnote{For brevity, structured SVM with slack-rescaling are not described in this document.}
can be formulated as a quadratic program (QP)
\begin{equation}
\label{eq:nslackform}
\begin{aligned}
\min_{\mathbf{w}, ~\bm{\xi} \ge 0} ~& \frac{1}{2} \mathbf{w}^\top \mathbf{w} + \frac{C}{n} \sum_{i=1}^n \xi_i \\
s.t.~~ ~& \mathbf{w}^\top \Psi(\mathbf{x}_i, \mathbf{y}_i) - \mathbf{w}^\top \Psi(\mathbf{x}_i, \bar{\mathbf{y}}) \ge 
       \Delta(\mathbf{y}_i, \bar{\mathbf{y}}) - \xi_i, ~(\forall i, \bar{\mathbf{y}} \neq \mathbf{y}_i)
\end{aligned}
\end{equation}
where $\mathbf{w}$ is the parameter vector, $C > 0$ is a regularisation constant, and $\xi_i$
is a slack variable that represents the \emph{hinge loss} associated with the prediction for the $i$-th example~\cite{tsochantaridis2005large},
\begin{equation*}
\xi_i = \max \left( 0,~ 
        \max_{\bar{\mathbf{y}} \in \mathcal{Y}} 
        \left\{ \Delta(\mathbf{y}_i, \bar{\mathbf{y}}) + \mathbf{w}^\top \Psi(\mathbf{x}_i, \bar{\mathbf{y}}) \right\} -
        \mathbf{w}^\top \Psi(\mathbf{x}_i, \mathbf{y}_i) \right).
\end{equation*}
This formulation is called "$n$-slack" as we have one slack variable for each example in training set. \eat{citation}

To train the $n$-slack formulation of structured SVM, one option is simply enumerating all constraints and 
solve optimisation problem (\ref{eq:nslackform}) using a standard QP solver, 
however, this approach is impractical as there is a constraint for every incorrect label $\bar{\mathbf{y}}$.
Instead, we use a cutting-plane algorithm that repeatedly solves QP (\ref{eq:nslackform}) with respect to different set of constraints, 
and each iteration generates a new constraint that helps reduce the feasible region of the problem, 
until a specified precision $\varepsilon$ is achieved~\cite{joachims2009predicting}, as described in Algorithm~\ref{alg:nslacktrain}.

\begin{algorithm}[htbp]
\caption{Cutting-plane algorithm for training $n$-slack formulation of structured SVM (with margin-rescaling)}
\label{alg:nslacktrain}
\begin{algorithmic}[1]
\STATE \textbf{Input}: $S = \left( (\mathbf{x}_1, \mathbf{y}_1), \dots, (\mathbf{x}_n, \mathbf{y}_n) \right),~ C,~ \varepsilon$
\STATE $\mathcal{W} = \emptyset,~ k = 1,~ \mathbf{w}^{(k)} = \mathbf{0},~ \bm{\xi}^{(k)} = \mathbf{0}$
\REPEAT
    \FOR{$i = 1,\dots,n$}
        \STATE $\triangleright$ Query the oracle at point $q^{(k)} = (\mathbf{w}^{(k)}, \bm{\xi}^{(k)})$ as follows
        \STATE Do loss-augmented inference:~
               $\hat{\mathbf{y}} = \argmax_{\bar{\mathbf{y}} \in \mathcal{Y}} \{ \Delta(\mathbf{y}_i, \bar{\mathbf{y}}) + 
                \langle \mathbf{w}^{(k)},~ \Psi(\mathbf{x}_i, \bar{\mathbf{y}}) \rangle \}$ 
        \IF{~$q^{(k)}$ is not feasible:~ $\langle \mathbf{w}^{(k)},~ \Psi(\mathbf{x}_i, \mathbf{y}_i) - \Psi(\mathbf{x}_i, \hat{\mathbf{y}}) \rangle + 
             \varepsilon < \Delta(\mathbf{y}_i, \hat{\mathbf{y}}) - \xi_i^{(k)}$~}
            \STATE Form a \emph{feasibility cut} and update constraints:~
                   $\mathcal{W} = \mathcal{W} \cup 
                    \left\{ \langle \mathbf{w},~ \Psi(\mathbf{x}_i, \mathbf{y}_i) - \Psi(\mathbf{x}_i, \hat{\mathbf{y}}) \rangle \ge 
                    \Delta(\mathbf{y}_i, \hat{\mathbf{y}}) - \xi_i \right\}$ 
            \STATE Generate the next query point $q^{(k+1)} = (\mathbf{w}^{(k+1)}, \bm{\xi}^{(k+1)})$ 
                   by solving QP~(\ref{eq:nslackform}) w.r.t. all constraints in $\mathcal{W}$
            \STATE $k = k+1$
        \ENDIF
    \ENDFOR
%\UNTIL{$\mathcal{W}$ has not changed during iteration}
\UNTIL{$q^{(k)}$ is feasible for all training examples}
\RETURN $q^{(k)}$
\end{algorithmic}
\end{algorithm}


\subsection{Train the $1$-slack formulation of structured SVM}
\label{sec:1slackssvm}

Another formulation of structured SVM which results in more efficient training is called "$1$-slack" formulation (with margin-rescaling),
it replaces the $n$ cutting-plane models of the hinge loss (one for each training example) with a single cutting-plane model for 
the sum of the hinge-losses~\cite{joachims2009cutting}, as a result, only one slack variable is needed,
\begin{equation}
\label{eq:1slackform}
\begin{aligned}
\min_{\mathbf{w}, ~\xi \ge 0} ~& \frac{1}{2} \mathbf{w}^\top \mathbf{w} + C \xi \\
s.t.~~ ~& \forall(\bar{\mathbf{y}}_1, \dots, \bar{\mathbf{y}}_n) \in \mathcal{Y}^n: 
          \frac{1}{n} \sum_{i=1}^n 
          \left( \mathbf{w}^\top \Psi(\mathbf{x}_i, \mathbf{y}_i) - \mathbf{w}^\top \Psi(\mathbf{x}_i, \bar{\mathbf{y}}_i) \right) \ge
          \frac{1}{n} \sum_{i=1}^n \Delta(\mathbf{y}_i, \bar{\mathbf{y}}_i) - \xi.
\end{aligned}
\end{equation}
Here the slack variable $\xi$ represents the \emph{sum of the hinge-losses} over all training examples,
\begin{equation*}
\xi = \max \left( 0,~ 
      \max_{(\bar{\mathbf{y}}_1, \dots, \bar{\mathbf{y}}_n) \in \mathcal{Y}^n} 
      \left\{ 
      \frac{1}{n} \sum_{i=1}^n \left( \Delta(\mathbf{y}_i, \bar{\mathbf{y}}_i) + \mathbf{w}^\top \Psi(\mathbf{x}_i, \bar{\mathbf{y}}_i) \right)
      \right\} - \frac{1}{n} \sum_{i=1}^n \mathbf{w}^\top \Psi(\mathbf{x}_i, \mathbf{y}_i)
      \right).
\end{equation*}


Compared with the $n$-slack formulation described in Section~\ref{sec:nslackssvm}, 
the $1$-slack formulation of structured SVM increases the number of constraints exponentially~\cite{joachims2009cutting},
which means enumerating all constraints is also impractical.
Algorithm~\ref{alg:1slacktrain} described an approach similar to Algorithm~\ref{alg:nslacktrain} that uses a cutting-plane method to 
train the $1$-slack formulation of structured SVM.


\begin{algorithm}[htbp]
\caption{Cutting-plane algorithm for training $1$-slack formulation of structured SVM (with margin-rescaling)}
\label{alg:1slacktrain}
\begin{algorithmic}[1]
\STATE \textbf{Input}: $S = \left( (\mathbf{x}_1, \mathbf{y}_1), \dots, (\mathbf{x}_n, \mathbf{y}_n) \right),~ C,~ \varepsilon$
\STATE $\mathcal{W} = \emptyset$
%\REPEAT
\FOR{$k = 1,\dots,+\infty$}
    \STATE Generate query point $q^{(k)} = (\mathbf{w}^{(k)}, \xi^{(k)})$ by solving QP~(\ref{eq:1slackform}) w.r.t. all constraints in $\mathcal{W}$
    \STATE $\triangleright$ Query the oracle at point $q^{(k)}$ as follows
    \STATE Do loss-augmented inference:~
           $\hat{\mathbf{y}}_i = \argmax_{\bar{\mathbf{y}} \in \mathcal{Y}} \left\{ \Delta(\mathbf{y}_i, \bar{\mathbf{y}}) + 
            \langle \mathbf{w}^{(k)},~ \Psi(\mathbf{x}_i, \bar{\mathbf{y}}) \rangle \right\},~ \forall i$
    \IF{~$q^{(k)}$ is $\varepsilon$-feasible:~ $\frac{1}{n} \sum_{i=1}^n 
         \langle \mathbf{w}^{(k)},~ \Psi(\mathbf{x}_i, \mathbf{y}_i) - \Psi(\mathbf{x}_i, \hat{\mathbf{y}}_i) \rangle + \varepsilon \ge 
         \frac{1}{n} \sum_{i=1}^n \Delta(\mathbf{y}_i, \hat{\mathbf{y}}_i) - \xi^{(k)}$~}
        \RETURN $q^{(k)}$
    \ELSE
        \STATE Form a \emph{feasibility cut} and update constraints:~
               $\mathcal{W} = \mathcal{W} \cup \left\{ 
                \frac{1}{n} \sum_{i=1}^n \langle \mathbf{w},~ \Psi(\mathbf{x}_i, \mathbf{y}_i) - \Psi(\mathbf{x}_i, \hat{\mathbf{y}}_i) \rangle \ge 
                \frac{1}{n} \sum_{i=1}^n \Delta(\mathbf{y}_i, \hat{\mathbf{y}}_i) - \xi \right\}$
    \ENDIF
%\UNTIL{$\frac{1}{n} \sum_{i=1}^n 
%        \left( \mathbf{w}^\top \Psi(\mathbf{x}_i, \mathbf{y}_i) - \mathbf{w}^\top \Psi(\mathbf{x}_i, \hat{\mathbf{y}}_i) \right) + 
%        \varepsilon \ge \frac{1}{n} \sum_{i=1}^n \Delta(\mathbf{y}_i, \hat{\mathbf{y}}_i) - \xi$}
%\RETURN $(\mathbf{w}, \xi)$
\ENDFOR
\end{algorithmic}
\end{algorithm}


\subsection{Discussion}
\label{sec:ssvm_discussion}

From Algorithm~\ref{alg:nslacktrain} and Algorithm~\ref{alg:1slacktrain}, we observe that:
\begin{itemize}
\item To generate a query point $q$, it solves a QP with the same objective as the original optimisation problem and
      all constraints/cuts returned by previous queries. 
\item The Wolfe-dual programs of both QP (\ref{eq:nslackform}) and QP (\ref{eq:1slackform}) are QPs~\cite{tsochantaridis2005large,joachims2009cutting}.
\item All cutting-planes returned by the oracle are \emph{feasibility cuts}.
\item The training algorithm will \emph{stop} if the current query point $q$ is feasible, 
      in other words, it does not exploit an \emph{objective cut} when $q$ is feasible. 
\end{itemize}


\subsubsection{Objective cut}
\label{sec:ssvm_objcut}

Recall that in Section~\ref{sec:problem}, we have an objective $f_0(z)$ to minimise, in the case of $1$-slack formulation of structured SVM,
$f_0(z)$ is the quadratic objective in Equation~(\ref{eq:1slackform}), 
\begin{equation}
\label{eq:optobj}
f_0(z) = \frac{1}{2} \mathbf{w}^\top \mathbf{w} + C\xi,
\end{equation}
where $z = [\mathbf{w}, \xi]^\top$.
Given query point $q = \left[ \mathbf{w}^{(k)}, \xi^{(k)} \right]^\top$, if $q$ is feasible, we can form an \emph{objective cut}
\begin{equation}
\label{eq:objcut_ssvm}
\begin{aligned}
 & \nabla f_0(q)^\top (z - q) \\
=& \left[ \left[ \left.\frac{\partial f_0}{\mathbf{w}}\right|_{\mathbf{w} = \mathbf{w}^{(k)}}, 
                 \left.\frac{\partial f_0}{\xi}\right|_{\xi = \xi^{(k)}} \right]^\top \right]^\top 
   \left( \left[ \mathbf{w}, \xi \right]^\top - \left[ \mathbf{w}^{(k)}, \xi^{(k)} \right]^\top \right)  \\
=& \left[ \mathbf{w}^{(k)}, C \right] \left[ \mathbf{w} - \mathbf{w}^{(k)},~ \xi - \xi^{(k)} \right]^\top  \\
=& \langle \mathbf{w}^{(k)},~ \mathbf{w} - \mathbf{w}^{(k)} \rangle + C (\xi - \xi^{(k)}) \le 0.
\end{aligned}
\end{equation}


\subsubsection{Feasibility cut}
\label{sec:ssvm_feacut}

On the other hand, if $q = (\mathbf{w}^{(k)}, \xi^{(k)})$ is not feasible, the following constraint must be violated,
\begin{equation}
\label{eq:cut_1slackssvm}
\frac{1}{n} \sum_{i=1}^n \langle \mathbf{w},~ \Psi(\mathbf{x}_i, \mathbf{y}_i) - \Psi(\mathbf{x}_i, \hat{\mathbf{y}}_i) \rangle \ge 
\frac{1}{n} \sum_{i=1}^n \Delta(\mathbf{y}_i, \hat{\mathbf{y}}_i) - \xi.
\end{equation}

Let 
\begin{equation}
\label{eq:constraint_k}
f_k(z) = \frac{1}{n} \sum_{i=1}^n \Delta(\mathbf{y}_i, \hat{\mathbf{y}}_i) - 
         \frac{1}{n} \sum_{i=1}^n \langle \mathbf{w},~ \Psi(\mathbf{x}_i, \mathbf{y}_i) - \Psi(\mathbf{x}_i, \hat{\mathbf{y}}_i) \rangle - \xi,
\end{equation}
where $z = [\mathbf{w}, \xi]^\top$.
We can rewrite the violated constraint as $f_k(z) \le 0$.
Since $q$ violates constraint (\ref{eq:cut_1slackssvm}), we can construct a \emph{feasibility cut}
\begin{equation}
\label{eq:feacut_ssvm}
\begin{aligned}
 & f_k(q) + \nabla f_k(q)^\top (z - q) \\
=& f_k(q) + 
   \left[ \left[ \left.\frac{\partial f_k}{\mathbf{w}}\right|_{\mathbf{w} = \mathbf{w}^{(k)}}, 
                 \left.\frac{\partial f_k}{\xi}\right|_{\xi = \xi^{(k)}} \right]^\top \right]^\top 
   \left( \left[ \mathbf{w}, \xi \right]^\top - \left[ \mathbf{w}^{(k)}, \xi^{(k)} \right]^\top \right)  \\
=& f_k(q) + \left[ -\frac{1}{n} \sum_{i=1}^n \left( \Psi(\mathbf{x}_i, \mathbf{y}_i) - \Psi(\mathbf{x}_i, \hat{\mathbf{y}}_i) \right),  -1 \right] 
   \left[ \mathbf{w} - \mathbf{w}^{(k)},~ \xi - \xi^{(k)} \right]^\top  \\
=& \frac{1}{n} \sum_{i=1}^n \Delta(\mathbf{y}_i, \hat{\mathbf{y}}_i) - 
   \frac{1}{n} \sum_{i=1}^n \langle \mathbf{w}^{(k)},~ \Psi(\mathbf{x}_i, \mathbf{y}_i) - \Psi(\mathbf{x}_i, \hat{\mathbf{y}}_i) \rangle - \xi^{(k)} +  
   \langle -\frac{1}{n} \sum_{i=1}^n \left( \Psi(\mathbf{x}_i, \mathbf{y}_i) - \Psi(\mathbf{x}_i, \hat{\mathbf{y}}_i) \right),~
   \mathbf{w} - \mathbf{w}^{(k)} \rangle - \left( \xi - \xi^{(k)} \right)  \\
=& \frac{1}{n} \sum_{i=1}^n \Delta(\mathbf{y}_i, \hat{\mathbf{y}}_i) - 
   \frac{1}{n} \sum_{i=1}^n \langle \mathbf{w},~ \Psi(\mathbf{x}_i, \mathbf{y}_i) - \Psi(\mathbf{x}_i, \hat{\mathbf{y}}_i) \rangle - \xi \le 0.
\end{aligned}
\end{equation}

We found that inequalities (\ref{eq:cut_1slackssvm}) and (\ref{eq:feacut_ssvm}) are identical.
This is \emph{not unexpected} as the plane tangent to $f_k(z)$ (also a plane) at point $q$ is \emph{identical} to plane $f_k(z)$
(assuming the same domain for $z$).
%which means that the cutting-planes Algorithm~\ref{alg:1slacktrain} exploited are exactly \emph{feasibility cuts}. 
%Note that this result does not dependent on the fact that 
%$\left( \hat{\mathbf{y}}_1, \dots, \hat{\mathbf{y}}_n \right)$ are the most violated labels (w.r.t. constraints in QP \ref{eq:1slackform}),
%which suggests that any labels $\left( \bar{\mathbf{y}}_1, \dots, \bar{\mathbf{y}}_n \right)$ that violate constraints in problem (\ref{eq:1slackform})
%can be used to form a feasibility cut.


\subsection{Compare with the method of Kelley-Cheney-Goldstein and the Chebyshev center method}

Given query points $q^{(1)}, \dots, q^{(k)}$ and the feasibility cuts returned by oracle (after querying these points), then
\begin{equation}
\label{eq:ssvm_kcg_cons}
\begin{aligned}
 & f_0(q^{(k)}) + \nabla f_0(q^{(k)})^\top (z - q^{(k)}) \\
=& \frac{1}{2} \langle \mathbf{w}^{(k)},~ \mathbf{w}^{(k)} \rangle + C\xi^{(k)} + 
   \left[ \mathbf{w}^{(k)}, C \right] \left[ \mathbf{w} - \mathbf{w}^{(k)},~ \xi - \xi^{(k)} \right]^\top  \\
=& \langle \mathbf{w}^{(k)}, \mathbf{w} \rangle - \frac{1}{2} \langle \mathbf{w}^{(k)}, \mathbf{w}^{(k)} \rangle + C\xi,
\end{aligned}
\end{equation}
where $q^{(k)} = (\mathbf{w}^{(k)}, \xi^{(k)})$ and $f_0(\cdot)$ is defined in Equation~(\ref{eq:optobj}).

If we use the method of Kelley-Cheney-Goldstein described in Section~\ref{sec:kcg} to generate the next query point $q^{(k+1)}$,
we need to solve the following optimisation problem,
\begin{equation}
\label{eq:ssvm_kcg}
\begin{aligned}
\min_{z} ~& \theta \\
s.t.~~ ~& \theta \ge \langle \mathbf{w}^{(k)}, \mathbf{w} \rangle - \frac{1}{2} \langle \mathbf{w}^{(k)}, \mathbf{w}^{(k)} \rangle + C\xi,~ \forall k \\
        & f_k(z) \le 0,~ \forall k, \\
        & -\xi \le 0,
\end{aligned}
\end{equation}
where $z = [\mathbf{w}, \xi]^\top$ and $f_k(z)$ is defined in Equation~(\ref{eq:constraint_k}).

We solve a similar optimisation problem if we use the Chebyshev center method described in Section~\ref{sec:chebyshev} to generate the next query point,
\begin{equation}
\label{eq:ssvm_chebyshev}
\begin{aligned}
\min_{z} ~& \theta \\
s.t.~~ ~& \theta \ge 
          \frac{1}{D_k} \langle \mathbf{w}^{(k)}, \mathbf{w} \rangle - 
          (\frac{1}{2} - \frac{1}{D_k}) \langle \mathbf{w}^{(k)}, \mathbf{w}^{(k)} \rangle + 
          \frac{C}{D_k}\xi + C (1 - \frac{1}{D_k}) \xi^{(k)},~ \forall k \\
        & f_k(z) \le 0,~ \forall k, \\
        & -\xi \le 0,
\end{aligned}
\end{equation}
where $D_k = \|\nabla f_0(q^{(k)})\| = \sqrt{\langle \mathbf{w}^{(k)}, \mathbf{w}^{(k)} \rangle + C^2}$ is a normalisation constant.


The method to generate the next query point used in Algorithm~\ref{alg:1slacktrain} can be rewritten as
\begin{equation}
\label{eq:ssvm_genquery}
\begin{aligned}
\min_{z} ~& \frac{1}{2} \mathbf{w}^\top \mathbf{w} + C \xi \\
s.t.~~ ~& f_k(z) \le 0,~ \forall k, \\
        & -\xi \le 0.
\end{aligned}
\end{equation}


Since $f_k(z)$ is a linear function, we know that both problem (\ref{eq:ssvm_kcg}) and (\ref{eq:ssvm_chebyshev}) are linear programs (LP),
and problem (\ref{eq:ssvm_genquery}) is a quadratic program (QP). 

\bibliographystyle{ieeetr}
\bibliography{ref}

\end{document}
