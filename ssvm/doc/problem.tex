\section{Problem formulation}
\label{sec:formulation}

The trajectory recommendation problem is: given a set of points-of-interest (POI) $\mathcal{P}$ and a trajectory query $\mathbf{x} = (s, K)$,
where $s \in \mathcal{P}$ is the desired start POI and $K > 1$ is the number of POIs in the desired trajectory (including the start location $s$).
We want to recommend a sequence of POIs $\mathbf{y}^*$ that maximises utility, i.e., for a suitable function $f(\cdot,\cdot)$,
\begin{equation*}
\mathbf{y}^* = \argmax_{\mathbf{y} \in \mathcal{Y}_\mathbf{x}}~f(\mathbf{x}, \mathbf{y}),
\end{equation*}
where $\mathcal{Y}_\mathbf{x}$ is the set of all possible trajectories with POIs in $\mathcal{P}$ and satisfying query $\mathbf{x}$.
$\mathbf{y} = (y_1 = s,~ y_2, \dots, y_K)$ is a trajectory with $K$ POIs, and $y_j \ne y_k$ if $j \ne k$ 
which is known as \emph{no duplicates constraint}.

Instead of the number of desired POIs, we can constrain the trajectory with a total time budget $T$.
In this case, the number of POIs $K$ can be treated as a \emph{hidden} variable, with additional constraint $\sum_{k=1}^K t_k \le T$ 
where $t_k$ is the time spent at POI $y_k$.



\subsection{A concrete example}
\label{sec:example}

Given a set of $10$ points-of-interest (POI) in Melbourne 
\begin{align*}
\mathcal{P} = \{ 
& \textit{\small Eureka Tower, Federation Square, Flinders Street Railway Station, Luna Park, Melbourne Aquarium, Melbourne Cricket Ground,} \\
& \textit{\small Melbourne Zoo, National Gallery of Victoria, Royal Exhibition Building, University of Melbourne} \}
\end{align*}
and a query $\mathbf{x} = \{\textit{\small University of Melbourne},~ 5\}$,
we would like to recommend a trajectory 
\begin{equation*}
\mathbf{y} = \{\textit{\small University of Melbourne},~ y_2, \dots, y_5\},~ y_k \in \mathcal{P},~ k=2,\dots,5.
\end{equation*}
by modelling trajectory data we have with POI and query related features as described in Section~\ref{sec:feature}.



\subsection{Related problems}
\label{sec:related}

This problem is related to automatic playlist generation, 
where we recommend a sequence of songs given a specified song (a.k.a. the seed) and the number of new songs.
Formally, given a library of songs and a query $\mathbf{x} = (s, K)$, where $s$ is the seed and $K$ is the number of songs in playlist,
we produce a list with $K$ songs (without duplication) by maximising the likelihood~\cite{chen2012playlist},
\begin{equation*}
%\max_{(y_1,\dots,y_K)} \prod_{k=2}^K \mathbb{P}(y_{k-1} \given y_k),~ y_1 = s ~\text{and}~ y_j \ne y_k,~ j \ne k.
\mathbf{y}^* = \argmax_{\mathbf{y} \in \mathcal{P}_\mathbf{x}}~ \mathbb{P}(\mathbf{y} \given \mathbf{x}),~ \mathbf{y} = (y_1=s,\dots,y_K) 
~\text{and}~ y_j \ne y_k ~\text{if}~ j \ne k.
\end{equation*}

Another similar problem is choosing a small set of photos from a large photo library and compiling them into a slideshow or movie.



\subsection{Evaluation metrics and loss functions}
\label{sec:evaluation}

To evaluate the performance of a certain recommendation algorithm,
we need to measure the similarity (or loss) given prediction $\hat{\mathbf{y}}$ and ground truth $\mathbf{y}$.
Metrics researchers have used include
\begin{itemize}
\item Hamming loss $\frac{1}{K} \sum_{j=1}^K \llb \hat{y}_j \neq y_j \rrb$, this checks if every position is the same.

\item F$_1$ score on points~\cite{ijcai15}, where we care about the set of correctly recommended POIs. 
      Let $\texttt{set}(\mathbf{y})$ denote the set of POIs in trajectory $\mathbf{y}$, F$_1$ score on points is defined as
\begin{equation*}
F_1 = \frac{2  P_{\textsc{point}}  R_{\textsc{point}}}{P_{\textsc{point}} + R_{\textsc{point}}} ~~\text{where}~
P_{\textsc{point}} = \frac{| \texttt{set}(\hat{\mathbf{y}}) \cap \texttt{set}(\mathbf{y}) |}{| \texttt{set}(\hat{\mathbf{y}}) |}~\text{and}~
R_{\textsc{point}} = \frac{| \texttt{set}(\hat{\mathbf{y}}) \cap \texttt{set}(\mathbf{y}) |}{| \texttt{set}(\mathbf{y}) |}.
\end{equation*}
If $| \hat{\mathbf{y}} | = | \mathbf{y} |$, this metric is just the unordered Hamming loss, 
i.e., Hamming loss between two binary indicator vectors of size $| \mathcal{P} |$.


\item F$_1$ score on pairs~\cite{cikm16paper}, where we care about the set of correctly predicted POI pairs,
\begin{equation*}
\text{pairs-F}_1 = \frac{2 P_{\textsc{pair}} R_{\textsc{pair}}}{P_{\textsc{pair}} + R_{\textsc{pair}}}~~\text{where}~
P_{\textsc{pair}} = \frac{N_c} {| \texttt{set}(\hat{\mathbf{y}}) | (| \texttt{set}(\hat{\mathbf{y}}) | - 1) / 2}~\text{and}~
R_{\textsc{pair}} = \frac{N_c} {| \texttt{set}(\mathbf{y}) | (| \texttt{set}(\mathbf{y}) | - 1) / 2},
\end{equation*}
and $N_c = \sum_{j=1}^{| \mathbf{y} | - 1} \sum_{k=j+1}^{| \mathbf{y} |} \llb y_j \prec_{\bar{\mathbf{y}}} y_k \rrb$,
here $y_j \prec_{\bar{\mathbf{y}}} y_k$ denotes that POI $y_j$ appears before POI $y_k$ in trajectory $\bar{\mathbf{y}}$.
We define pairs-F$_1 = 0$ when $N_c = 0$.

\end{itemize}

However, if we cast a trajectory $\mathbf{y} = (y_1,\dots,y_K)$ as a ranking of POIs in $\mathcal{P}$,
where $y_k$ has a rank $| \mathcal{P} | - k + 1$ and any other POI $p \notin \mathbf{y}$ has a rank $0$ ($0$ is an arbitrary choice).
We can make use of ranking evaluation metrics such as Kendall's $\tau$ or Spearman's $\rho$, by taking care of ties in ranks.

\eat{TODO: Write these down and contrast, esp. to pairs-F1}.

