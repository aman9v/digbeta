\documentclass[9pt]{extarticle}
\usepackage[a4paper,top=0.79in,left=0.79in,bottom=0.79in,right=0.79in]{geometry} % A4 paper margins in LibreOffice
\usepackage{hyperref}
\usepackage{amsmath}
\usepackage{amsfonts}
\usepackage{stmaryrd}
\usepackage[sc]{mathpazo}
\linespread{1.05}         % Palladio needs more leading (space between lines)
\usepackage[T1]{fontenc}
\usepackage{multirow}
\usepackage{array}

\title{Features}
\author{}
\date{}

\begin{document}
\maketitle

\begin{table}[ht]
\caption{List of "session" features used to model the behavior of a tourist in a city~\cite{baraglia2013learnext}.}
\centering
\begin{tabular}{l|m{.6\textwidth}} \hline
\textbf{Feature Name} & \textbf{Description} \\ \hline
actualTransferTime & Total transfer time from a POI to the next one in a session. \\ \hline
actualVisitTime    & The visit time for all POIs in a session. \\ \hline
categsPerSess      & Number of categories per session. \\ \hline
distLat\_Avg       & \multirow{8}{*}{Average, Max, Min, Total Latitude and Longitude distance between POIs in a session.} \\
distLat\_Max       & \\
distLat\_Min       & \\
distLat\_Tot       & \\
distLen\_Avg       & \\
distLen\_Max       & \\
distLen\_Min       & \\
distLen\_Tot       & \\ \hline
euclideanDist\_Avg & \multirow{4}{*}{Average, Max, Min, Total Euclidean distance between POIs in a session.} \\
euclideanDist\_Max & \\
euclideanDist\_Min & \\
euclideanDist\_Total & \\ \hline
phPOISess\_Avg     & \multirow{4}{*}{Average, Max, Min, Total number of photos of POIs in a session.} \\
phPOISess\_Max     & \\
phPOISess\_Min     & \\
phPOISess\_Tot     & \\ \hline
uniqueCategsPerSess & The number of unique categories per session. \\ \hline
sessLen            & Number of POIs in a session. \\ \hline
sessTime           & Total time for a session, from beginning to end. \\ \hline
userSessLen\_Avg   & \multirow{4}{*}{Average, Max, Min, Total length of sessions belonging to a user.} \\
userSessLen\_Max   & \\
userSessLen\_Min   & \\
userSessLen\_Total & \\ \hline
userSessRatio      & The ratio between the number of sessions made by the user and the maximum number of sessions for a user.\\ \hline
\end{tabular}
\end{table}


\begin{table}[ht]
\caption{List of "POI" features used to model the characteristics of each candidate destination~\cite{baraglia2013learnext}.}
\centering
\begin{tabular}{l|m{.6\textwidth}} \hline
\textbf{Feature Name}  & \textbf{Description} \\ \hline
cat1, cat2, ..., cat10 & Top 10 most frequent categories.\\ \hline
distFromFirstPOI\_Eucl & \multirow{6}{*}{Latitude, Longitude and Euclidean distance from last and first POI of the session.} \\
distFromFirstPOI\_Lat  & \\
distFromFirstPOI\_Len  & \\
distFromLastPOI\_Eucl  & \\
distFromLastPOI\_Lat   & \\
distFromLastPOI\_Len   & \\ \hline
entropy                & The entropy of the last POI in the session. \\ \hline
freqBigrams            & The frequency of the POI given the last POI in session.\\ \hline
freqTrigrams           & The frequency of the POI given the last two POIs in session. \\ \hline
middleProbab           & The probability that the POI is within a trail and not in the extremes. \\ \hline
numCategories          & The number of categories assigned to the POI. \\ \hline
numPhotos\_Avg         & \multirow{4}{*}{Average, Max, Min and Total number of photos of the POI in the collection.} \\ 
numPhotos\_Max         & \\
numPhotos\_Min         & \\
numPhotos\_Total       & \\ \hline
noOfVisits             & The total number of visits of a POI in the collection.\\ \hline
photosPerUser          & The total number of photos of belonging to a user. \\ \hline
photosPOI\_userId\_Avg   & \multirow{2}{*}{Average and total number of photos of a POI for a user.} \\
photosPOI\_userId\_Total & \\ \hline
ratioPhotosPOI           & The ratio between the number of photos for the POI and the maximum number of photos for a POI. \\ \hline
ratioPOIInUserPhotos     & The ratio between the number of photos of a POI for a user and all the photos belonging to the user. \\ \hline
ratioSessWithPOI         & The ratio between the number of sessions containing the POI and the total number of sessions. \\ \hline
ratioUsersVisitingPOI    & The ratio between the number of users visiting the POI and the total number of users. \\ \hline
startProb                & \multirow{2}{*}{The probability that a POI is first or last in a trail.} \\  
stopProb                 & \\ \hline
visitTimePOI\_User       & The total visit time of a POI for a user. \\ \hline
visitTime\_Avg           & \multirow{5}{*}{Average, Max, Min, StdDev and Total visit time of the POI.} \\
visitTime\_Max           & \\
visitTime\_Min           & \\ 
visitTime\_StdDev        & \\
visitTime\_Total         & \\ \hline
\end{tabular}
\end{table}


POI popularity, visiting time, categorization.
Each POI $p$ is identified by its geographic coordinates, a name, a radius specifying its spatial extent, 
and a relevance vector, $v_p \in [0, 1]^{|C|}$, measuring the normalized relevance of $p$ w.r.t a set of categories $C$. 
we assume that the set $C$ is predetermined and fixed and that the relevance of every POI for each category is known~\cite{brilhante2013shall}.


POIs for each city C are obtained from various sources, including Yahoo! Travel (\texttt{http://travel.yahoo.com/}) and 
Lonely Planet (\texttt{http://travel.lonelyplanet.com/}).
Each POI is then described with the following attributes: \textit{pname} uniquely identifies the POI;
\textit{city} is the city it belongs to; and $g_{lat}^l$ and $g_{long}^l$ are its latitude and longitude. 
Examples including museums, parks, historical sites, and religious places~\cite{ht10}.


\begin{table}[ht]
\caption{Description of the venue-dependent Foursquare features (Venue) used in this work~\cite{deveaud2014importance}.}
\centering
\begin{tabular}{l|m{.6\textwidth}} \hline
\textbf{Feature name} & \textbf{Description} \\ \hline
NbCheckins & Total number of check-ins in the venue. \\ \hline
NbLikes    & Total number of "likes" for the venue. \\ \hline
NbTips     & Total number of "tips" for the venue. \\ \hline
NbPhotos   & Total number of photos that have been taken in the venue. \\ \hline
Rating     & Average of all the ratings given by the users for the venue. \\ \hline
CheckinRatio & $\frac{\text{NbCheckins}}{\text{NbCheckinsInCity}}$ \\ \hline
LikeRatio    & $\frac{\text{NbLikes}}{\text{NbLikesInCity}}$ \\ \hline
TipRatio     & $\frac{\text{NbTips}}{\text{NbTipsInCity}}$ \\ \hline
PhotoRatio   & $\frac{\text{NbPhotos}}{\text{NbPhotosInCity}}$ \\ \hline
Distance     & Distance of the venue from the center of the city. \\ \hline
\end{tabular}
\end{table}


Each POI $p$ is also labelled with a category $Cat_p$ (e.g., church, park, beach) and latitude/longitude coordinates. 
We denote a function $Pop(p)$ that indicates the popularity of a POI $p$, based on the number of times POI $p$ has been visited~\cite{ijcai15}.

POI category tree from Foursquare, Gowalla; user preference transition over POI categories, category-aware POI recommendation~\cite{liu2013personalized}.

Modelling geographical neighborhood of a location. The neighborhood is modelled at two levels:~\cite{liu2014exploiting}
\begin{itemize}
\item the instance-level neighborhood defined by a few nearest neighbors of the location, 
      i.e., nearest neighboring locations tend to share more similar user preferences
\item the region-level neighborhood for the geographical region where the location exists. 
      i.e., locations in the same geographical region may share similar user preferences
\end{itemize}

\bibliographystyle{ieeetr}
\bibliography{ref}

\end{document}
