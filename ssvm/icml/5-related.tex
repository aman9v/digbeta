\section{Related work}
\label{sec:related}

This note attempts to relate trajectory recommendation to several problems in the ranking literature.
At present, we believe the problem bears similarity to both subset and label ranking.

%
\textbf{Trajectory recommendation}.
Let $\CCal$ be a set of ``points of interest'' (POIs) in a city.
Let $\QCal = \CCal \times \mathbb{N}_+$ be a set of possible queries, which comprise a start POI and a trip length.
%For any $n \in \mathbb{N}_+$,
%let
%$\YCal_n = \CCal^{n}$ be the set of all possible length $n$-sequences of POIs,
%which we call trajectories of length $n$.
Let
$\YCal = \bigcup_{n = 1}^\infty \CCal^{n}$ be the set of all possible sequences of POIs, which we call trajectories.
In trajectory recommendation,
we have a training sample $\SSf = \{ ( q^{(i)}, y^{(i)} ) \}_{i = 1}^N$,
where
$q^{(i)} = ( s^{(i)}, n^{(i)} ) \in \QCal$,
and
%$y^{(i)} \in \YCal_{n^{(i)}}$
$y^{(i)} \in \YCal$
is a trajectory of the specified length.
Our goal is to learn a trajectory recommender %$f \colon \QCal \to \bigcup_{n = 1}^\infty \YCal_n$.
$f \colon \QCal \to \YCal$.

In general, $\SSf$ is characterised by the same $q \in \QCal$ appearing multiple times, with different trajectories $y \in \YCal$.
One can equally partition the training sample by query, yielding
$ \SSf = \{ ( q^{(j)}, ( y^{(1)}, \ldots, y^{(N_j)} ) ) \}_{j = 1}^{M} $,
where $M$ is the number of unique queries,
and $N_j$ is the number of trajectories associated with the $j$th query.

We want to relate this problem to extant problems in the learning to rank literature.
There are two main strands of work here:
one where the ranking is over \emph{instances},
another where the ranking is over \emph{labels}.
We survey representative problems from each of these settings,
and connect the ``master'' problem from each to trajectory recommendation.

\begin{table*}[!h]
	\centering
	\renewcommand{\arraystretch}{1.25}
	\begin{tabular}{@{}llll@{}}
		\toprule
		\toprule
		\textbf{Problem} & \textbf{Input} & \textbf{Output} & \textbf{Comments} \\
		\toprule
		Trajectory recommendation & $\{ ( q^{(i)}, y^{(i)} ) \}$ 		 & $f \colon \QCal \to \YCal$ & $\YCal = \bigcup_{n = 1}^\infty [K]^n$ \\
		\midrule
		Instance ranking & $\{ ( x^{(i)}, y^{(i)} ) \}$ 		 & $r \colon \XCal \to \Real$ & $\YCal$ totally ordered \\
		Subset ranking 	 & $\{ ( q^{(i)}, x^{(i)}, y^{(i)} ) \}$ & $r \colon \QCal \times \XCal \to \Real$ & $\YCal$ totally ordered \\
		\midrule
		Multilabel learning & $\{ ( x^{(i)}, y^{(i)} ) \}$ 		 & $f \colon \XCal \to \YCal$ & $\YCal = \{ 0, 1 \}^K$ \\
		Label ranking 		& $\{ ( x^{(i)}, y^{(i)} ) \}$ 		 & $r \colon \XCal \times [K] \to \Real$ & $\YCal = \{ 0, 1 \}^{K \times K}$ \\
		\bottomrule
	\end{tabular}
	
	\caption{Ranking problems considered in document; we have (instance, label, query) sets $(\XCal,\YCal,\QCal)$, and categories $K \in \mathbb{N}_+$.}
	\label{tbl:problems}
\end{table*}

\vspace{12pt}
\hrule
\vspace{12pt}

%
\textbf{Multipartite instance ranking}.
Pick any instance set $\XCal$.
For some $C \in \mathbb{N}_+$,
let $\YCal = \{ 0, 1, \ldots, C - 1 \}$.
In multipartite instance ranking,
we have a training sample $\SSf = \{ ( x^{(i)}, y^{(i)} ) \}_{i = 1}^N$,
where
$x^{(i)} \in \XCal$,
$y^{(i)} \in \YCal$.
Our goal is to learn an instance ranker $r \colon \XCal \to \Real$.
Performance is measured using the pairwise disagreement,
$$ \mathrm{PD}( r ) = \ES{\X, \Y, \Y'}{ \indicator{\Y > \Y'} \cdot \indicator{ r( \X ) < r( \X' ) } }. $$

When $C = 2$, $\YCal = \{ 0, 1 \}$, and this is the bipartite ranking problem.

\emph{Example}.
Suppose $\XCal$ represents features of various movies, and $C$ is the number of possible star ratings a person can give a movie.
Then, we are given a snapshot of a single person's preference ranking over movies,
where movies with higher star ratings are ranked above those with lower star ratings.
We would look to find the complete preference ranking $r \colon \XCal \to \Real$ over movies.


%
\textbf{Subset ranking}.
Pick any instance set $\XCal$,
query set $\QCal$,
and totally ordered label set $\YCal$.
In subset ranking,\footnote{This appears to just be called ``learning to rank'', \eg in Tie-Yan Liu's book, and elsewhere in information retrieval. In machine learning, the terminology appears to come from a 2008 paper of Cossock and Zhang.}
we have a training sample $\SSf = \{ ( q^{(i)}, x^{(i)}, y^{(i)} ) \}_{i = 1}^N$,
where
$q^{(i)} \in \QCal$,
$x^{(i)} \in \XCal$,
$y^{(i)} \in \YCal$.
Our goal is to learn a query-instance ranker $r \colon \QCal \times \XCal \to \Real$.
Performance is measured using the
query-aggregated
pairwise disagreement,
$$ \mathrm{PD}_{\mathrm{query}}( r ) = \E{\Q}{ \ES{\X, \Y, \Y' \mid \Q}{ \indicator{ \Y > \Y' } \cdot \indicator{ r( \Q, \X ) < r( \Q, \X' ) } } }. $$

This captures multipartite ranking as a special case.
Suppose $\QCal = \{ 1 \}$, so that every example has the same query.
Then, the problem trivially reduces to the multipartite setting.

In general, $\SSf$ is characterised by the same $q \in \QCal$ appearing multiple times, with different trajectories $y \in \YCal$.
One can equally partition the training sample by query, yielding
$ \SSf = \{ ( q^{(j)}, ( ( x^{(1)}, y^{(1)} ), \ldots, ( x^{(N_j)}, y^{(N_j)} ) ) ) \}_{j = 1}^{M} $,
where $M$ is the number of unique queries,
and $N_j$ is the number of instance-label pairs associated with the $j$th query.

\emph{Example}.
Suppose $\XCal$ represents features of various movies, $\YCal = \{ 0, 1, \ldots, C - 1 \}$ where $C$ is the number of possible star ratings a person can give a movie,
and $\QCal$ represents features of various people.
Then, in subset ranking we are given snapshots of \emph{multiple} persons' preference rankings over movies.
We would look to find the complete preference ranking $r \colon \QCal \times \XCal \to \Real$ for \emph{any} possible user.

\emph{Reduction from trajectory recommendation?}.
Suppose $\XCal = \emptyset$,
and let $\QCal, \YCal$ be as in the trajectory recommendation problem.
Then, the training samples in the trajectory recommendation and subset ranking problem are identical.
However, the subset ranking problem requires that we be able to compare two elements of $\YCal$, \ie that we can order the set of all trajectories of a given length.
It is not obvious how to do this.

Alternately, suppose $\XCal = \CCal$,
let $\QCal$ be as in the trajectory recommendation problem,
and let $\YCal = \{ 0, 1 \}$.
Then, for a given query in trajectory recommendation,
the corresponding trajectories could be unrolled into the POIs that are present in the trajectory, with an associated positive label;
the POIs that are not present could be associated with a negative label.

One could similarly do this for pairs of POIs, where the label is whether one POI appears ahead of another.
This bears similarity to the trajectory recommendation problem when using pairwise features.


\vspace{12pt}
\hrule
\vspace{12pt}

%
\textbf{Multilabel learning}.
Pick any instance set $\XCal$
and category set $\CCal$.
%For some $C \in \mathbb{N}_+$,
Let $\YCal = \{ 0, 1 \}^{|\CCal|}$.
In multilabel learning,
we have a training sample $\SSf = \{ ( x^{(i)}, y^{(i)} ) \}_{i = 1}^N$,
where
$x^{(i)} \in \XCal$,
$y^{(i)} \in \YCal$.
Our goal is to learn a multilabel classifier $f \colon \XCal \to \YCal$.
Performance is measured via the risk
$$ R( f ) = \E{\X, \Y}{ \ell( \Y, f( \X ) ) } $$
for some multilabel loss function $\ell \colon \YCal \times \YCal \to \Real_+$.
Examples are the subset 0-1 loss
$$ \ell_{\mathrm{sub}}( y, \hat{y} ) = \indicator{ y \neq \hat{y} }, $$
Hamming loss
$$ \ell_{\mathrm{hamm}}( y, \hat{y} ) = \sum_{c = 1}^{|\CCal|} \indicator{ y_c \neq \hat{y}_c }, $$
and ranking loss
$$ \ell_{\mathrm{rank}}( y, \hat{y} ) = \sum_{c, c' = 1}^{|\CCal|} \indicator{ y_c < y_{c'} } \cdot \indicator{ \hat{y}_c > \hat{y}_c }. $$

\emph{Example}.
Suppose $\XCal$ represents features of various movies, and
$\CCal = \{ 0, 1, \ldots, U \}$ where $U$ is some number of people who use a rating system.
Then, in multilabel learning, we are given a snapshot of people's binary preference for movies.
We would look to find a predictor of these people's preferences for any given movie.


%
\textbf{Label ranking}.
Pick any instance set $\XCal$
and category set $\CCal$.
Let $\YCal \subset \{ 0, 1 \}^{|\CCal| \times |\CCal|}$ represent the set of (adjacency matrices for) all possible graphs over $\CCal$, excluding those with self-loops.
In label ranking,
we have a training sample $\SSf = \{ ( x^{(i)}, y^{(i)} ) \}_{i = 1}^N$,
where
$x^{(i)} \in \XCal$,
$y^{(i)} \in \YCal$.
Our goal is to learn a label ranker $r \colon \XCal \times \CCal \to \Real$.
Performance is measured using the disagreement error
$$ \mathrm{DA}( r ) = \E{\X, \Y}{ \frac{1}{\sum_{c, c'} \Y_{c c'}} \cdot \sum_{c, c' = 1}^{C} \indicator{\Y_{c c'} = 1} \cdot \indicator{ r( \X, c ) < r( \X, c' ) } }. $$

This captures multilabel learning as a special case.
Suppose $\YCal$ comprises the set of bipartite graphs.
Then, we can equally encode any $y \in \YCal$
as a vector $y \in \{ 0, 1 \}^{|\CCal|}$,
where $y_i = 1$ if and only if $i$ is a source node in the graph.

\emph{Example}.
...

\emph{Reduction from trajectory recommendation?}.
Let $\CCal$ be a set of POIs above.
Then $\YCal$ is the set of possible graphs over POIs,
which includes the set of sequence graphs.
If we treat the query set $\QCal$ in the trajectory recommendation problem as the instance set $\XCal$ in the label ranking problem,
then the training samples for both problems align.
That is, we can view trajectory recommendation as a special case of label ranking.

One subtlety is the issue of the same query in the trajectory recommendation problem having multiple associated ground truths. 
This is of course allowed in the label ranking problem (\ie it does not break the formulation),
but is not explicitly dealt with it while training.
In trajectory recommendation, one uses a ``query normalised'' training objective.

\vspace{12pt}
\hrule
\vspace{12pt}

{\color{red} TODO: probabilistic models of rankings, \eg Mallows, Plackett-Luce.}

{\color{red} TODO: write down the objective explicitly.}

{\color{red} TODO: multi-output multi-class classification.}
