\section{Multiple Ground Truths}

We consider the problem of supervised learning with multiple ground truths. In many practical
problems we may observe more than one label for the same set of features, which violates
the implicit assumptions of many learning algorithms. In this work we explicitly consider
all observed labels of a particular example to be useful for training, that is we use
the multiset of ground truths in training.
In particular we focus on the structured prediction case,
where the output of the classifier is from a large set $\mathcal{Y}$ with internal structure.
An example of this is when $y\in\mathcal{Y}$ is a sequence of binary values.
Given an example $x_i$ there may be multiple label sequences $y_{ij}$, where $j=1,...,J$.

Suggested order:
\begin{enumerate}
  \item structured SVM
  \item multiset SSVM
  \item list Viterbi for multiple ground truths
\end{enumerate}

Then focus on trajectory
\begin{enumerate}
  \item Trajectory recommendation
  \item ILP for subtour elimination
  \item 2 uses of list Viterbi
  \begin{itemize}
    \item multiple ground truths
    \item subtour elimination
  \end{itemize}
\end{enumerate}


\section{Trajectory recommendation}
\label{sec:formulation}

The trajectory recommendation problem is: given a set of points-of-interest (POI) $\mathcal{P}$ and a trajectory query $\mathbf{x} = (s, K)$,
where $s \in \mathcal{P}$ is the desired start POI and $K > 1$ is the number of POIs in the desired trajectory (including the start location $s$).
We want to recommend a sequence of POIs $\mathbf{y}^*$ that maximises utility, i.e., for a suitable function $f(\cdot,\cdot)$,
\begin{equation*}
\mathbf{y}^* = \argmax_{\mathbf{y} \in \mathcal{Y}_\mathbf{x}}~f(\mathbf{x}, \mathbf{y}),
\end{equation*}
where $\mathcal{Y}_\mathbf{x}$ is the set of all possible trajectories with POIs in $\mathcal{P}$ and satisfying query $\mathbf{x}$.
$\mathbf{y} = (y_1 = s,~ y_2, \dots, y_K)$ is a trajectory with $K$ POIs, and $y_j \ne y_k$ if $j \ne k$
which is known as \emph{no duplicates constraint}.

Instead of the number of desired POIs, we can constrain the trajectory with a total time budget $T$.
In this case, the number of POIs $K$ can be treated as a \emph{hidden} variable, with additional constraint $\sum_{k=1}^K t_k \le T$
where $t_k$ is the time spent at POI $y_k$.



\subsection{Related problems}
\label{sec:related}

This problem is related to automatic playlist generation,
where we recommend a sequence of songs given a specified song (a.k.a. the seed) and the number of new songs.
Formally, given a library of songs and a query $\mathbf{x} = (s, K)$, where $s$ is the seed and $K$ is the number of songs in playlist,
we produce a list with $K$ songs (without duplication) by maximising the likelihood~\cite{chen2012playlist},
\begin{equation*}
%\max_{(y_1,\dots,y_K)} \prod_{k=2}^K \mathbb{P}(y_{k-1} \given y_k),~ y_1 = s ~\text{and}~ y_j \ne y_k,~ j \ne k.
\mathbf{y}^* = \argmax_{\mathbf{y} \in \mathcal{P}_\mathbf{x}}~ \mathbb{P}(\mathbf{y} \given \mathbf{x}),~ \mathbf{y} = (y_1=s,\dots,y_K)
~\text{and}~ y_j \ne y_k ~\text{if}~ j \ne k.
\end{equation*}

Another similar problem is choosing a small set of photos from a large photo library and compiling them into a slideshow or movie.
