\section{Trajectory Recommendation as Structured Prediction}
\label{sec:trajrec}

The trajectory recommendation problem is: given a set of points-of-interest (POI) $\mathcal{P}$ and a trajectory query $\mathbf{x} = (s, K)$,
where $s \in \mathcal{P}$ is the desired start POI and $K > 1$ is the number of POIs in the desired trajectory (including the start location $s$).
We want to recommend a sequence of POIs $\mathbf{y}^*$ that maximises utility, i.e., for a suitable function $f(\cdot,\cdot)$,
\begin{equation*}
\mathbf{y}^* = \argmax_{\mathbf{y} \in \mathcal{Y}}~f(\mathbf{x}, \mathbf{y}),
\end{equation*}
where $\mathcal{Y}$ is the set of all possible trajectories with POIs in $\mathcal{P}$ and conform to query $\mathbf{x}$.
$\mathbf{y} = (y_1 = s,~ y_2, \dots, y_K)$ is a trajectory with $K$ POIs, and $y_j \ne y_k$ if $j \ne k$, 
i.e.,
there is no sub-tours in trajectory $\mathbf{y}$.


Instead of specifying the number of desired POIs, we can constrain the trajectory with a total time budget $T$.
In this case, the number of POIs $K$ can be treated as a \emph{hidden} variable, with additional constraint $\sum_{k=1}^K t_k \le T$
where $t_k$ is the time spent at POI $y_k$.


This problem is related to automatic playlist generation,
where we recommend a sequence of songs given a specified song (a.k.a. the seed) and the number of new songs.
Formally, given a library of songs and a query $\mathbf{x} = (s, K)$, where $s$ is the seed and $K$ is the number of songs in playlist,
we produce a list with $K$ songs (without duplication) by maximising the likelihood~\cite{chen2012playlist},
\begin{equation*}
%\max_{(y_1,\dots,y_K)} \prod_{k=2}^K \mathbb{P}(y_{k-1} \given y_k),~ y_1 = s ~\text{and}~ y_j \ne y_k,~ j \ne k.
\mathbf{y}^* = \argmax_{\mathbf{y} \in \mathcal{P}_\mathbf{x}}~ \mathbb{P}(\mathbf{y} \given \mathbf{x}),~ \mathbf{y} = (y_1=s,\dots,y_K)
~\text{and}~ y_j \ne y_k ~\text{if}~ j \ne k.
\end{equation*}

Another similar problem is choosing a small set of photos from a large photo library and compiling them into a slideshow or movie.


% standard SSVM
We can learn a recommender by training a SSVM on the set of observed trajectories $\{\mathbf{x}^{(i')}, \mathbf{y}^{(i')}\}_{i'=1}^{N'}$,
However, we ignore the fact that for the same query, we normally observed more than one trajectory,
we would like to exploit this fact to better modelling the observed trajectories.


\subsection{Query Aggregation}
\label{sec:query}

To modelling the fact that a given query has multiple observed trajectories, 
we firstly group trajectories according to queries, in other words,
we now have a dataset $\{\mathbf{x}^{(i)}, \{\mathbf{y}^{(ij)}\}_{j=1}^{N_i}\}_{i=1}^N$
with $N$ queries and queries $\mathbf{x}^{(i)}$ has $N_i$ trajectories observed.


\subsection{Recommendation with Multiset SSVM}
\label{sec:trajrec-ssvm}

We can learn to recommend trajectories by training a multiset SSVM described in Section~\ref{sec:ssvm-ms}

% multiset SSVM

% multiset SSVM: training, prediction

% sub-tour elimination
\section{Sub-tour Elimination}
\label{sec:subtour}

Both the loss-augmented inference and prediction inference for structured SVM (Section~\ref{sec:ssvm}) cannot be done efficiently 
if the no duplicates constraints are required, moreover, we would like to recommend more than one trajectories given a query.
To achieve this, we resort to a variant of the list Viterbi algorithm~\cite{nilsson2001sequentially,seshadri1994list}
which sequentially find the $k$-th best (scored) trajectory given the best, $2$nd best, \dots, $(k-1)$-th best (scored) trajectories,
as described in Algorithm~\ref{alg:listviterbi}.


% ILP for subtour elimination
The no duplicates constraint can be achieved by adapting the sub-tour elimination constraints of the travelling salesman problem (TSP),
in particular, we solve the following integer linear program (ILP),
\begin{alignat}{5}
& \max_{u,v} ~&& \sum_{k=1}^M \mathbf{w}_k^\top \phi_k(\mathbf{x}, p_k) \sum_{j=1}^M u_{jk} + 
                 \sum_{j=1}^M \sum_{k=1}^M u_{jk} \mathbf{w}_{jk}^\top \phi_{j, k}(\mathbf{x}, p_j, p_k) \\
& s.t. ~~ ~&& u_{jk}, ~z_j \in \{0, 1\}, ~u_{jj}=0, ~z_1=0, ~v_j \in \mathbf{Z},~ p_j \in \mathcal{P}, ~\forall j, k = 1,\cdots,M   \label{eq:cons1} \\
&          && \sum_{k=2}^M u_{1k} = 1, ~\sum_{j=2}^M u_{j1} = 0  \label{eq:cons2} \\
&          && \sum_{j=1}^M u_{jl} = z_l + \sum_{k=2}^M u_{lk} \le 1,   ~\forall l=2,\cdots,M                    \label{eq:cons3} \\
&          && \sum_{j=1}^M \sum_{k=1}^M u_{jk} = L-1,                                                           \label{eq:cons4} \\
&          && v_j - v_k + 1 \le (M-1) (1-u_{jk}),                     \forall j,k=2,\cdots,M                    \label{eq:cons5}
\end{alignat}
where $u_{jk}$ is a binary decision variable that determines whether the transition from $p_j$ to $p_k$ is in the resulting trajectory,
$z_j$ is a binary decision variable that determines whether $p_j$ is the last POI in trajectory.
$L$ is the number of POIs in trajectory.
For brevity, we arrange the POIs such that $p_1 = s$.
Firstly, the desired trajectory should start from $s$ (Constraint~\ref{eq:cons2}).
In addition, any POI could be visited at most once (Constraint~\ref{eq:cons3}).
Moreover, only $L-1$ transitions between POIs are permitted (Constraint~\ref{eq:cons4}),
i.e., the number of POI visits should be exactly $L$ (including $s$).
The last constraint, where $v_i$ is an auxiliary variable,
enforces that only a single sequence of POIs without sub-tours is permitted in the trajectory.

If we employ the above ILP to do loss-augmented inference, one restriction is that the loss should be a linear function of $u_{jk}$,
e.g., $\Delta(\mathbf{y}, \bar{\mathbf{y}}) = 1 - \sum_{j=1}^M \sum_{k=1}^M u_{j, y_k}$ if we define the loss as the number of mispredicted POIs,
where $\mathbf{y}$ is the ground truth and $\bar{\mathbf{y}}$ is the trajectory corresponding to the optimal solution of this ILP.


% 2 uses of list Viterbi: 1) multiple ground truths; 2) subtour elimination
