% Use the following line _only_ if you're still using LaTeX 2.09.
%\documentstyle[icml2017,epsf,natbib]{article}
% If you rely on Latex2e packages, like most moden people use this:
\documentclass{article}

% use Times
\usepackage{times}
% For figures
\usepackage{graphicx} % more modern
%\usepackage{epsfig} % less modern
\usepackage{subfigure}

% For citations
\usepackage{natbib}

% For algorithms
\usepackage{algorithm}
\usepackage{algorithmic}

% As of 2011, we use the hyperref package to produce hyperlinks in the
% resulting PDF.  If this breaks your system, please commend out the
% following usepackage line and replace \usepackage{icml2017} with
% \usepackage[nohyperref]{icml2017} above.
\usepackage{hyperref}

% Packages hyperref and algorithmic misbehave sometimes.  We can fix
% this with the following command.
\newcommand{\theHalgorithm}{\arabic{algorithm}}

% Employ the following version of the ``usepackage'' statement for
% submitting the draft version of the paper for review.  This will set
% the note in the first column to ``Under review.  Do not distribute.''
\usepackage{icml2017}

% Employ this version of the ``usepackage'' statement after the paper has
% been accepted, when creating the final version.  This will set the
% note in the first column to ``Proceedings of the...''
%\usepackage[accepted]{icml2017}


%%%%%%%%%%%%%%%%%%%%%%%%%%%%%%%%%
% Custom packages
%%%%%%%%%%%%%%%%%%%%%%%%%%%%%%%%%
% AKM: already present above!
%\usepackage{hyperref}
\usepackage{amsmath}
\usepackage{amsfonts}
\usepackage{mathrsfs}
\usepackage{bm}
\usepackage{bbm}
\usepackage{stmaryrd}
\usepackage{algorithm}
\usepackage{algorithmic}
%\usepackage[sc]{mathpazo}
%\linespread{1.05}         % Palladio needs more leading (space between lines)
\usepackage[T1]{fontenc}    % use 8-bit T1 fonts
%\usepackage{subcaption}  % sub-figures
\usepackage{nicefrac}       % compact symbols for 1/2, etc.
\usepackage{microtype}      % microtypography
\usepackage{booktabs} % professional-quality tables
\graphicspath{{fig/}} % Location of the graphics files

\usepackage{enumitem}

%%%%%%%%%%%%%%%%%%%%%%%%%%%%%%%%%
% Custom notation
%%%%%%%%%%%%%%%%%%%%%%%%%%%%%%%%%
\DeclareMathOperator*{\argmin}{argmin}
\DeclareMathOperator*{\argmax}{argmax}
\newcommand{\given}{\mid}
\newcommand{\llb}{\llbracket}
\newcommand{\rrb}{\rrbracket}
\newcommand{\indicator}[1]{\llbracket #1 \rrbracket}
\newcommand{\eat}[1]{}
\newcommand{\Q}{\mathsf{Q}}
\newcommand{\X}{\mathsf{X}}
\newcommand{\Y}{\mathsf{Y}}
\newcommand{\CCal}{\mathscr{C}}
\newcommand{\QCal}{\mathscr{Q}}
\newcommand{\XCal}{\mathscr{X}}
\newcommand{\YCal}{\mathscr{Y}}
\newcommand{\SSf}{\mathsf{S}}
\newcommand{\Real}{\mathbb{R}}
\newcommand{\E}[2]{\underset{#1}{\mathbb{E}}\left[ #2 \right]}
\newcommand{\ES}[2]{\underset{#1}{\mathbb{E}}\, #2}

% madeness: suPer-script in Brackets
\newcommand{\pb}[1]{^{({#1})}}

\newcommand{\x}{\mathbf{x}}
\newcommand{\y}{\mathbf{y}}
\newcommand{\w}{\mathbf{w}}

\newcommand{\eg}{e.g.\ }
\newcommand{\ie}{i.e.\ }

\newcommand{\TODO}[1]{ {\color{blue}{\bf TODO:~{#1}}} }



% The \icmltitle you define below is probably too long as a header.
% Therefore, a short form for the running title is supplied here:
\icmltitlerunning{Structured Recommendation}

\begin{document}

\twocolumn[
\icmltitle{Sequence Recommendation with Structured Prediction}
%\icmltitle{Structured Recommendation}

% It is OKAY to include author information, even for blind
% submissions: the style file will automatically remove it for you
% unless you've provided the [accepted] option to the icml2017
% package.

% list of affiliations. the first argument should be a (short)
% identifier you will use later to specify author affiliations
% Academic affiliations should list Department, University, City, Region, Country
% Industry affiliations should list Company, City, Region, Country

% you can specify symbols, otherwise they are numbered in order
% ideally, you should not use this facility. affiliations will be numbered
% in order of appearance and this is the preferred way.
\icmlsetsymbol{equal}{*}

\begin{icmlauthorlist}
\icmlauthor{Aeiau Zzzz}{equal,to}
\icmlauthor{Bauiu C.~Yyyy}{equal,to,goo}
\icmlauthor{Cieua Vvvvv}{goo}
\icmlauthor{Iaesut Saoeu}{ed}
\icmlauthor{Fiuea Rrrr}{to}
\icmlauthor{Tateu H.~Yasehe}{ed,to,goo}
\icmlauthor{Aaoeu Iasoh}{goo}
\icmlauthor{Buiui Eueu}{ed}
\icmlauthor{Aeuia Zzzz}{ed}
\icmlauthor{Bieea C.~Yyyy}{to,goo}
\icmlauthor{Teoau Xxxx}{ed}
\icmlauthor{Eee Pppp}{ed}
\end{icmlauthorlist}

\icmlaffiliation{to}{University of Torontoland, Torontoland, Canada}
\icmlaffiliation{goo}{Googol ShallowMind, New London, Michigan, USA}
\icmlaffiliation{ed}{University of Edenborrow, Edenborrow, United Kingdom}

\icmlcorrespondingauthor{Cieua Vvvvv}{c.vvvvv@googol.com}
\icmlcorrespondingauthor{Eee Pppp}{ep@eden.co.uk}

% You may provide any keywords that you
% find helpful for describing your paper; these are used to populate
% the "keywords" metadata in the PDF but will not be shown in the document
\icmlkeywords{structured prediction, sequence recommendation}

\vskip 0.3in
]

% this must go after the closing bracket ] following \twocolumn[ ...

% This command actually creates the footnote in the first column
% listing the affiliations and the copyright notice.
% The command takes one argument, which is text to display at the start of the footnote.
% The \icmlEqualContribution command is standard text for equal contribution.
% Remove it (just {}) if you do not need this facility.

\printAffiliationsAndNotice{}  % leave blank if no need to mention equal contribution
%\printAffiliationsAndNotice{\icmlEqualContribution} % otherwise use the standard text.
%\footnotetext{hi}

\begin{abstract}
Content recommendation systems have seen considerable success, but have largely focussed on the case of static, unstructured content.
In many recommendation scenarios, we would like to predict content that has some sequential structure,
such as when recommending a playlist of songs, or a trajectory of points-of-interest in a city.
In this paper, we present an approach to such sequential recommendation problems based on techniques from structured prediction.
While the problem can be posed as a vanilla application of such models,
we propose two important corrections that are necessary to ensure quality recommendations.
First, we modify the training objective to take into account the existence of multiple ground truths.
Second, we modify the training and inference procedures to avoid predicting loops in the sequence via an extension of the classic Viterbi algorithm.
Experiments on two real-world trajectory recommendation datasets shows the benefits of our approach over existing, non-structured recommendation approaches.
\end{abstract}

Contributions
\begin{itemize}
  \item multiset SSVM
  \item two uses of list Viterbi
  \item apply to trajectory recommendation
  \item better than CIKM2016
  \item framework to understand related work
\end{itemize}


% !TEX root=./main.tex

\section{Related work}
\label{sec:related}

We contrast the problem setting and approach of this paper to a number of sub-fields in the recommender systems and machine learning literature.

%
\subsection{Recommender systems}

There is a rich body of work on recommender systems,
with the 
Netflix prize~\citep{Netflix} encapsulating the canonical problem that concerns most research in this field:
the recommendation of \emph{static} content such as books or movies from some large database~\citep{Goldberg:1992,Resnick:1994,Konstan:1997,Sarwar:2001,Koren:2010}.
Matrix factorisation methods have proven particularly effective for such problems~\citep{Koren:2009}.
The latter are a prototypical example of a \emph{collaborative filtering} approach to recommendation,
wherein one exploits the ``wisdom of the crowd'' implicit in the preferences of several users.

Standard matrix factorisation methods
have been extended to tackle
diverse problem settings such as
time-varying preferences~\citep{Koren:2009b} and implicitly provided feedback~\citep{Hu:2008,Rendle:2009}.
However,
the setting of recommending \emph{structured content},
such as trajectories as considered in this paper,
has received less attention but for a few exceptions (see below).
There are clear challenges with applying matrix factorisation techniques in our problem.
First, most users are only associated with a single trajectory, which defeats any hope of inferring a complex preference embedding for them.
Second, even if one had access to multiple trajectories per user, it is unclear how to find a latent embedding for entire trajectories,
as there are exponentially many possible values for the latter.


%
\subsection{Structured content recommendation}

The recommendation of structured content has
been studied in three distinct subfields.
The first is work on recommending the next item a user might like to purchase, given the sequence of their shopping basket purchases~\citep{Rendle:2010,Wang:2015}.
The canonical approach here is to apply the matrix factorisation idea to the Markov chain of transitions between items.
The success of this method relies on the fact that one is only interested in predicting sequences one element at a time, so that latent embeddings for each item may be found.

The second is work on recommending song playlists to users, given a query song~\citep{McFee:2011,chen2012playlist}.
The canonical approach here is to 
learn a latent representation of songs from historical playlist data,
and exploit a Markovian assumption on the song transitions.
While a reasonable first order approximation, this assumption limits the modelling power of such approaches.

The third is prior work on travel route recommendation, which we survey in detail.


%
\subsection{Travel route recommendation}

Recommendation problems involving travel routes have received considerable interest of late~\cite{bao2015recommendations,zheng2015trajectory,zheng2014urban}.
There are, roughly, three problem settings that have been studied.
The first setting is \emph{point of interest} (\emph{POI}) \emph{recommendation}.
Here, one simply wishes to rank various POIs in a city in order of how ``interesting'' they are to a given visitor,
exploiting
available metadata for each POI.
Typically, this problem is tackled via 
collaborative filtering on user-location affinity~\cite{shi2011personalized,lian2014geomf,liu2014exploiting,yuan2013timeaware,hsieh2014mining,gao2013temporal,yuan2014graph}.

The second setting is \emph{next location recommendation}.
Here, given the sequence of a traveller's partial tour through a city,
the goal is to recommend which POI the traveller should visit next.
This can be understood as a variant of POI recommendation with strong contextual information provided.
Typically, this problem is tackled via 
incorporating Markov chains into collaborative filtering~\cite{fpmc10,ijcai13,zhang2015location},
%quantifying tourist traffic flow between points-of-interest~\cite{zheng2012patterns},
%formulating a binary decision or ranking problem~\cite{baraglia2013learnext}, and predicting the next location with
or sequence models such as recurrent neural networks~\cite{aaai16}.

The third setting is trajectory recommendation,
which is our focus in this paper.
Typically, this problem is tackled via 
a heuristic combination of locations and routes~\cite{lu2010photo2trip,ijcai15,lu2012personalized}, or
by solving an optimisation problem that does not exploit historical data~\cite{gioniswsdm14,chen2015tripplanner}.


%
\subsection{Learning to rank}

% AKM: probably can omit this stuff as I dunno how to frame it properly

The trajectory recommendation problem can be related at an abstract level to the label ranking problem~\citep{Dekel:2003},
where the input comprises a query and a corresponding graph as a label.
The goal in such problems is to learn a ranking over the nodes of the graph,
which is similar to our setting;
however, these are typically not learned using structured prediction models.

Our problem can also be related to the listwise ranking approach in information retrieval~\citep{Cao:2007},
which attempts to learn a good set of results for a query by exploiting structure embedded in the entire set of results.
One contribution of this work is in casting trajectory recommendation as such a structured prediction problem, as opposed to
pointwise and pairwise approaches considered in previous work.


\section{Multiple Ground Truths}
\label{sec:multiset}

We consider the problem of supervised learning with multiple ground truths. In many practical
problems we may observe more than one label for the same set of features, which violates
the implicit assumptions of many learning algorithms. In this work we explicitly consider
all observed labels of a particular example to be useful for training, that is we use
the multiset of ground truths in training.
In particular we focus on the structured prediction case,
where the output of the classifier is from a large set $\mathcal{Y}$ with internal structure.
An example of this is when $y\in\mathcal{Y}$ is a sequence of binary values.
Given an example $x_i$ there may be multiple label sequences $y_{ij}$, where $j=1,...,J$.

\eat{
Suggested order:
\begin{enumerate}
  \item structured SVM
  \item multiset SSVM
  \item list Viterbi for multiple ground truths
\end{enumerate}

Then focus on trajectory
\begin{enumerate}
  \item Trajectory recommendation
  \item ILP for subtour elimination
  \item 2 uses of list Viterbi
  \begin{itemize}
    \item multiple ground truths
    \item subtour elimination
  \end{itemize}
\end{enumerate}
}

\subsection{Structured SVM}
\label{sec:ssvm}

In structured prediction, the output of classifier given feature vector $\mathbf{x}$ is
\begin{equation*}
\mathbf{y}^* = \argmax_{\mathbf{y} \in \mathcal{Y}}~ f(\mathbf{x}, \mathbf{y}),
\end{equation*}
%where $\mathcal{Y}_\mathbf{x}$ is the set of all possible trajectories with POIs in $\mathcal{P}$ and satisfying query $\mathbf{x}$,
where $f(\mathbf{x}, \mathbf{y})$ is a function that scores the compatibility between features $\mathbf{x}$ and a specific label $\mathbf{y}$,
in the case of structured SVM (SSVM), the compatibility function $f(\mathbf{x}, \mathbf{y})$ for structured SVM is this linear form,
\begin{equation*}
f(\mathbf{x}, \mathbf{y}) = \mathbf{w}^\top \Psi(\mathbf{x}, \mathbf{y}),
\end{equation*}
where $\Psi(\mathbf{x}, \mathbf{y})$ is a \emph{joint feature map} 
that captures features extracted from both $\mathbf{x}$ and label $\mathbf{y}$.

The design of joint feature $\Psi(\cdot,\cdot)$ is problem specific, 
for many problems, we can assume the joint feature be decomposed into singleton and pairwise interactions, i.e.,
\begin{equation*}
\label{eq:jointfeature}
\mathbf{w}^\top \Psi(\mathbf{x}, \mathbf{y}) 
= \sum_{j=2}^{| \mathbf{y} |} 
  \left( \mathbf{w}_j^\top \Psi_j(\mathbf{x}, y_j) + 
  \mathbf{w}_{j-1,j}^\top \Psi_{j-1, j}(\mathbf{x}, y_{j-1}, y_j) \right),
\end{equation*}
where $\Psi_j$ is a feature vector of singleton $y_j$ 
and $\Psi_{j-1,j}$ is a pairwise feature vector that captures the interactions between $y_{j-1}$ and POI $y_j$.

To learn the parameters, we train the structured SVM by optimising a quadratic program (QP),
\begin{equation}
\label{eq:nslack}
\begin{aligned}
\min_{\mathbf{w}, \, \bm{\xi} \ge 0} ~& \frac{1}{2} \mathbf{w}^\top \mathbf{w} + \frac{C}{N} \sum_{i=1}^N \xi_i \\
s.t.~ ~& \langle \mathbf{w}, \, \Psi(\mathbf{x}^{(i)}, \mathbf{y}^{(i)}) - \Psi(\mathbf{x}^{(i)}, \bar{\mathbf{y}}) \rangle \ge 
       \Delta(\mathbf{y}^{(i)}, \bar{\mathbf{y}}) - \xi_i, \, \forall i,
\end{aligned}
\end{equation}
where $\bar{\mathbf{y}} \in \mathcal{Y}$ and $\Delta(\mathbf{y}, \bar{\mathbf{y}})$ is a discrepancy function that measures the loss 
for predicting $\bar{\mathbf{y}}$ given ground truth $\mathbf{y}$, 
and slack variable $\xi_i$ is the \emph{hinge loss} for the prediction of the $i$-th example~\cite{tsochantaridis2005large},
\begin{equation*}
\xi_i = \max \left( 0, \,
        \max_{\bar{\mathbf{y}} \in \mathcal{Y}} 
        \left\{ \Delta(\mathbf{y}^{(i)}, \bar{\mathbf{y}}^{(i)}) + \mathbf{w}^\top \Psi(\mathbf{x}^{(i)}, \bar{\mathbf{y}}) \right\} -
        \mathbf{w}^\top \Psi(\mathbf{x}^{(i)}, \mathbf{y}^{(i)}) \right).
\end{equation*}
%This formulation is called "$n$-slack" as we have one slack variable for each example in training set.

We can rewrite the constraint in problem (\ref{eq:nslack}) as
\begin{equation}
\label{eq:ssvminf}
\mathbf{w}^\top \Psi(\mathbf{x}^{(i)}, \mathbf{y}^{(i)}) + \xi_i \ge
          \max_{\bar{\mathbf{y}} \in \mathcal{Y}}
          \left\{\mathbf{w}^\top \Psi(\mathbf{x}^{(i)}, \bar{\mathbf{y}}) + \Delta(\mathbf{y}^{(i)}, \bar{\mathbf{y}}) \right\},
\end{equation}
where the right hand side is known as the \emph{loss-augmented inference}.

To solve problem (\ref{eq:nslack}), one option is simply enumerating all constraints, and feeding the problem into a standard QP solver.
However, this approach is impractical as there is a constraint for every possible label $\bar{\mathbf{y}}$.
Instead, we use a cutting-plane algorithm which repeatedly solves QP (\ref{eq:nslack}) 
w.r.t. different set of constraints~\cite{joachims2009predicting}.
In each iteration, a new constraint is formed by solving the loss-augmented inference, 
which helps shrink the feasible region of the problem.


\subsection{Multiset SSVM}
\label{sec:ssvm-ms}

If we observed more than one labels for a particular set of features, 
the classic SSVM described in Section~\ref{sec:ssvm} can be generalised to capture the multiple ground truths setting,
in particular, given feature vector $\mathbf{x}^{(i)}$ and the corresponding set of ground truths $\{\mathbf{y}^{(ij)}\}_{j=1}^{n_i}$ 
where $n_i$ is the number of labels for $\mathbf{x}_i$,
we can train a multiset SSVM by optimising a QP similar to (\ref{eq:nslack}),
\begin{equation}
\label{eq:nslack_ml}
\begin{aligned}
\min_{\mathbf{w}, \, \bm{\xi} \ge 0} ~& \frac{1}{2} \mathbf{w}^\top \mathbf{w} + \frac{C}{N} \sum_{i,j} \xi_{ij} \\
s.t.~ ~& \langle \mathbf{w}, \, \Psi(\mathbf{x}^{(i)}, \mathbf{y}^{(ij)}) - \Psi(\mathbf{x}^{(i)}, \bar{\mathbf{y}}) \rangle \ge 
         \Delta(\mathbf{y}^{(ij)}, \bar{\mathbf{y}}) - \xi_{ij}, \, \forall j, \, \forall i.
\end{aligned}
\end{equation}
where $N = \sum_i n_i$ and $\bar{\mathbf{y}} \in \mathcal{Y} \setminus \{\mathbf{y}^{(ij)}\}_{j=1}^{n_i}$.



\subsection{Training Multiset SSVM}
\label{sec:train-ssvm-ms}
% list Viterbi for multiple ground truths

Similar to the loss-augmented inference described in Section~\ref{sec:ssvm}, 
we can rewrite the constraints in problem (\ref{eq:nslack_ml}) into
\begin{equation*}
\mathbf{w}^\top \Psi(\mathbf{x}^{(i)}, \mathbf{y}^{(ij)}) + \xi_{ij} \ge 
\max_{\bar{\mathbf{y}}} \left( \Delta(\mathbf{y}^{(ij)}, \bar{\mathbf{y}}) + \mathbf{w}^\top \Psi(\mathbf{x}^{(i)}, \bar{\mathbf{y}}) \right),
\, \forall j.
\end{equation*} 
Normally, the maximisation at the right side of the above inequality can not be solved efficiently due to the constraint that 
$\bar{\mathbf{y}}$ is in $\mathcal{Y}$ but should not be in the set of observed labels $\{\mathbf{y}^{(ij)}\}_{j=1}^{n_i}$.
However, we can pretend that $\bar{\mathbf{y}}$ can be any label in $\mathcal{Y}$ and do the unconstrained optimisation,
which can sometimes be solved efficiently, then we filtering out the optimal solution if it has been observed, 
i.e., in $\{\mathbf{y}^{(ij)}\}_{j=1}^{n_i}$. 
In addition to train multiset SSVM, this technique can be further used to deal with other constraints such as sub-tour elimination 
as described in Section~\ref{sec:subtour}.

\section{Application on Trajectory Recommendation}
\label{sec:trajrec}

The trajectory recommendation problem is: given a set of points-of-interest (POI) $\mathcal{P}$ and a trajectory query $\mathbf{x} = (s, K)$,
where $s \in \mathcal{P}$ is the desired start POI and $K > 1$ is the number of POIs in the desired trajectory (including the start location $s$).
We want to recommend a sequence of POIs $\mathbf{y}^*$ that maximises utility, i.e., for a suitable function $f(\cdot,\cdot)$,
\begin{equation*}
\mathbf{y}^* = \argmax_{\mathbf{y} \in \mathcal{Y}_\mathbf{x}}~f(\mathbf{x}, \mathbf{y}),
\end{equation*}
where $\mathcal{Y}_\mathbf{x}$ is the set of all possible trajectories with POIs in $\mathcal{P}$ and satisfying query $\mathbf{x}$.
$\mathbf{y} = (y_1 = s,~ y_2, \dots, y_K)$ is a trajectory with $K$ POIs, and $y_j \ne y_k$ if $j \ne k$
which is known as \emph{no duplicates constraint}.

Instead of the number of desired POIs, we can constrain the trajectory with a total time budget $T$.
In this case, the number of POIs $K$ can be treated as a \emph{hidden} variable, with additional constraint $\sum_{k=1}^K t_k \le T$
where $t_k$ is the time spent at POI $y_k$.


This problem is related to automatic playlist generation,
where we recommend a sequence of songs given a specified song (a.k.a. the seed) and the number of new songs.
Formally, given a library of songs and a query $\mathbf{x} = (s, K)$, where $s$ is the seed and $K$ is the number of songs in playlist,
we produce a list with $K$ songs (without duplication) by maximising the likelihood~\cite{chen2012playlist},
\begin{equation*}
%\max_{(y_1,\dots,y_K)} \prod_{k=2}^K \mathbb{P}(y_{k-1} \given y_k),~ y_1 = s ~\text{and}~ y_j \ne y_k,~ j \ne k.
\mathbf{y}^* = \argmax_{\mathbf{y} \in \mathcal{P}_\mathbf{x}}~ \mathbb{P}(\mathbf{y} \given \mathbf{x}),~ \mathbf{y} = (y_1=s,\dots,y_K)
~\text{and}~ y_j \ne y_k ~\text{if}~ j \ne k.
\end{equation*}

Another similar problem is choosing a small set of photos from a large photo library and compiling them into a slideshow or movie.

Both the loss-augmented inference and prediction inference for structured SVM (Section~\ref{sec:ssvm}) cannot be done efficiently 
if the no duplicates constraints are required, moreover, we would like to recommend more than one trajectories given a query.
To achieve this, we resort to a variant of the list Viterbi algorithm~\cite{nilsson2001sequentially,seshadri1994list}
which sequentially find the $k$-th best (scored) trajectory given the best, $2$nd best, \dots, $(k-1)$-th best (scored) trajectories,
as described in Algorithm~\ref{alg:listviterbi}.


\subsection{Sub-tour Elimination}
\label{sec:subtour}

% ILP for subtour elimination
The no duplicates constraint can be achieved by adapting the sub-tour elimination constraints of the travelling salesman problem (TSP),
in particular, we solve the following integer linear program (ILP),
\begin{alignat}{5}
& \max_{u,v} ~&& \sum_{k=1}^M \mathbf{w}_k^\top \phi_k(\mathbf{x}, p_k) \sum_{j=1}^M u_{jk} + 
                 \sum_{j=1}^M \sum_{k=1}^M u_{jk} \mathbf{w}_{jk}^\top \phi_{j, k}(\mathbf{x}, p_j, p_k) \\
& s.t. ~~ ~&& u_{jk}, ~z_j \in \{0, 1\}, ~u_{jj}=0, ~z_1=0, ~v_j \in \mathbf{Z},~ p_j \in \mathcal{P}, ~\forall j, k = 1,\cdots,M   \label{eq:cons1} \\
&          && \sum_{k=2}^M u_{1k} = 1, ~\sum_{j=2}^M u_{j1} = 0  \label{eq:cons2} \\
&          && \sum_{j=1}^M u_{jl} = z_l + \sum_{k=2}^M u_{lk} \le 1,   ~\forall l=2,\cdots,M                    \label{eq:cons3} \\
&          && \sum_{j=1}^M \sum_{k=1}^M u_{jk} = L-1,                                                           \label{eq:cons4} \\
&          && v_j - v_k + 1 \le (M-1) (1-u_{jk}),                     \forall j,k=2,\cdots,M                    \label{eq:cons5}
\end{alignat}
where $u_{jk}$ is a binary decision variable that determines whether the transition from $p_j$ to $p_k$ is in the resulting trajectory,
$z_j$ is a binary decision variable that determines whether $p_j$ is the last POI in trajectory.
$L$ is the number of POIs in trajectory.
For brevity, we arrange the POIs such that $p_1 = s$.
Firstly, the desired trajectory should start from $s$ (Constraint~\ref{eq:cons2}).
In addition, any POI could be visited at most once (Constraint~\ref{eq:cons3}).
Moreover, only $L-1$ transitions between POIs are permitted (Constraint~\ref{eq:cons4}),
i.e., the number of POI visits should be exactly $L$ (including $s$).
The last constraint, where $v_i$ is an auxiliary variable,
enforces that only a single sequence of POIs without sub-tours is permitted in the trajectory.

If we employ the above ILP to do loss-augmented inference, one restriction is that the loss should be a linear function of $u_{jk}$,
e.g., $\Delta(\mathbf{y}, \bar{\mathbf{y}}) = 1 - \sum_{j=1}^M \sum_{k=1}^M u_{j, y_k}$ if we define the loss as the number of mispredicted POIs,
where $\mathbf{y}$ is the ground truth and $\bar{\mathbf{y}}$ is the trajectory corresponding to the optimal solution of this ILP.


% 2 uses of list Viterbi: 1) multiple ground truths; 2) subtour elimination

\begin{algorithm}[htbp]
\caption{The list Viterbi algorithm for inference}
\label{alg:listviterbi}
\begin{algorithmic}[1]
\STATE \textbf{Input}: $\mathbf{x}=(s, K),~ \mathcal{P},~ \mathbf{w},~ \Psi,~ l$
%\STATE Initialise score matrices $\alpha,~ \beta,~ f_t,~ f_{t, t+1}$, a max-heap $H,~ k=0$.
\STATE Initialise score matrices $\alpha,~ \beta,~ f_{t, t+1}$, a max-heap $H$, result set $R$, $k=0$.
\STATE $\triangleright$ Do the forward-backward procedure~\cite{rabiner1989tutorial}
\STATE $\forall p_j \in \mathcal{P},~ \alpha_t(p_j) = 
        \begin{cases}
        0,~ t = 1 \\
        \max_{p_i \in \mathcal{P}} \left\{ \alpha_{t-1}(p_i) + \mathbf{w}_{ij}^\top \Psi_{ij}(\mathbf{x}, p_i, p_j) + 
        \mathbf{w}_j^\top \Psi_j(\mathbf{x}, p_j) \right\},~ t=2,\dots,K
        \end{cases}$

\STATE $\forall p_i \in \mathcal{P},~ \beta_t(p_i) = 
        \begin{cases}
        0,~ t = K \\
        \max_{p_j \in \mathcal{P}} \left\{ \mathbf{w}_{ij}^\top \Psi_{ij}(\mathbf{x}, p_i, p_j) + 
        \mathbf{w}_j^\top \Psi_j(\mathbf{x}, p_j) + \beta_{t+1}(p_j) \right\},~ t = K-1,\dots,1
        \end{cases}$

%\STATE $\forall p_i \in \mathcal{P},~ f_t(p_i) = \alpha_t(p_i) + \beta_t(p_i),~ t = 1,\dots,K$
\STATE $\forall p_i, p_j \in \mathcal{P},~ f_{t,t+1}(p_i, p_j) = \alpha_t(p_i) + \mathbf{w}_{ij}^\top \Psi_{ij}(\mathbf{x}, p_i, p_j) + 
                              \mathbf{w}_j^\top \Psi_j(\mathbf{x}, p_j) + \beta_{t+1}(p_j),~ t = 1,\dots,K-1$

\STATE $\triangleright$ Identify the best (scored) trajectory $\mathbf{y}^1=(y_1^1,\dots,y_K^1)$ (possibly with sub-tours)
\STATE $y_t^1 = \begin{cases}
                s,~ t = 1 \\
%                \argmax_{p \in \mathcal{P}} \left\{ f_{1,2}(s, p) \right\},~ t = 2, \\
                \argmax_{p \in \mathcal{P}} \left\{ f_{t-1,t}(y_{t-1}^1, p) \right\},~ t = 2,\dots,K
                \end{cases}$

%\STATE $r^1 = \max_{p \in \mathcal{P}} \left\{ f_K(p) \right\}~~~ \triangleright$ $r^1$ is the score/priority of $\mathbf{y}^1$
\STATE $r^1 = \max_{p \in \mathcal{P}} \left\{ \alpha_{K}(p) \right\}~~~ \triangleright$ $r^1$ is the score/priority of $\mathbf{y}^1$
\STATE $H.\textit{push}\left(r^1,~ (\mathbf{y}^1, \textsc{nil}, \emptyset) \right)$

\WHILE{$H \ne \emptyset$ \textbf{and} $k < \,|\mathcal{P}|^{K-1} - \prod_{t=2}^K (|\mathcal{P}|-t+1)$}
    \STATE $r^k,~ (\mathbf{y}^k, I, S) = H.\textit{pop}()~~~ \triangleright$ 
           $r^k$ is the score of $\mathbf{y}^k=(y_1^k,\dots,y_K^k)$, $I$ is the partition index, and $S$ is the exclude set
    \STATE $k = k + 1$
    \STATE Add $\mathbf{y}^k$ to $R$ if NO sub-tours in $\mathbf{y}^k$
    \RETURN $R$ if it contains $l$ trajectories
    \STATE $\bar{I} = \begin{cases}
                      2,~ I = \textsc{nil} \\
                      I,~ \text{otherwise}
                      \end{cases}$

    \FOR{$t = \bar{I},\dots,K$}
        \STATE $\bar{S} = \begin{cases}
                          S \cup \{ y_t^k \},~ t = \bar{I} \\
                          \{ y_t^k \},~ \text{otherwise}
                          \end{cases}$

        \STATE $\bar{y}_j = \begin{cases}
                            y_j^k,~~ j=1,\dots,t-1 \\
                            %\argmax_{p \in \mathcal{P} \setminus \textit{new\_exclude\_set}} f_{t-1,t}(y_{t-1}^k, p),~ j=t \\
                            \argmax_{p \in \mathcal{P} \setminus \bar{S}} \left\{ f_{t-1,t}(y_{t-1}^k, p) \right\},~ j=t \\
                            \argmax_{p \in \mathcal{P}} \left\{ f_{j-1, j}(\bar{y}_{j-1}, p) \right\},~ j=t+1,\dots,K
                \end{cases}$
        \STATE $\bar{r} = \begin{cases}
                          f_{t-1,t}(y_{t-1}^k, \bar{y}_t),~ I = \textsc{nil} \\
                          r^k + f_{t-1,t}(y_{t-1}^k, \bar{y}_t) - f_{t-1,t}(y_{t-1}^k, y_t^k),~ \text{otherwise}
                          \end{cases}$

        $H.\textit{push}\left(\bar{r}, (\bar{\mathbf{y}}, t, \bar{S}) \right)$
    \ENDFOR
\ENDWHILE
\end{algorithmic}
\end{algorithm}
 

\section{Experiment}
\label{sec:experiment}

\subsection{Dataset and Features}
\label{sec:dataset}

% experiment protocol: Nested cross-validation with Monte-Carlo cross-validation for the inner loop
We experiment methods developed in Section~\ref{sec:trajrec} on trajectories extracted from Flickr photos~\cite{thomee2016yfcc100m}.

\subsection{Features}
\label{sec:feature}

The POI and query specific features extracted from trajectories are shown in Table~\ref{tab:poifeature},
features that describe the transition preference between different POIs are shown in Table~\ref{tab:tranfeature}.

\begin{table*}[ht]
\caption{Features of POI $p$ with respect to query $(s,K)$}
\label{tab:poifeature}
\centering


\setlength{\tabcolsep}{10pt} % tweak the space between columns
\begin{tabular}{l|l} \hline
\textbf{Feature}       & \textbf{Description} \\ \hline
\texttt{category}      & one-hot encoding of the category of $p$ \\
\texttt{neighbourhood} & one-hot encoding of the POI cluster that $p$ resides in \\
\texttt{popularity}    & logarithm of POI popularity of $p$ \\
\texttt{nVisit}        & logarithm of the total number of visit by all users at $p$ \\
\texttt{avgDuration}  & logarithm of the average visit duration at $p$ \\
\hline
%\texttt{nOccurrence}            & the number of times $p$ occurred in a trajectory that satisfies the query \\ DON'T know given new query

\texttt{trajLen}           & trajectory length $K$, i.e., the number of POIs required \\
\texttt{sameCatStart}      & $1$ if the category of $p$ is the same as that of $s$, $-1$ otherwise \\
\texttt{sameNeighbourhoodStart} & $1$ if $p$ resides in the same POI cluster as $s$, $-1$ otherwise \\
\texttt{diffPopStart}    & real-valued difference in POI popularity of $p$ from that of $s$ \\
\texttt{diffNVisitStart}        & real-valued difference in the total number of visit at $p$ from that at $s$ \\
\texttt{diffDurationStart}  & real-valued difference in average duration at $p$ from that at $s$ \\
\texttt{distStart}          & distance between $p$ and $s$, calculated using the Haversine formula \\
\hline
\end{tabular}
\end{table*}



\begin{table}[ht]
\caption{POI features used to estimate the (feature-wise) transition probabilities}
\label{tab:tranfeature}
\centering
\setlength{\tabcolsep}{2pt} % tweak the space between columns
\begin{tabular}{l|l} \hline
\textbf{Feature}       & \textbf{Description} \\ \hline
\texttt{category}      & category of POI \\
\texttt{neighbourhood} & the cluster that a POI resides in \\
\texttt{popularity}    & (discretised) popularity of POI \\
\texttt{nVisit}        & (discretised) total number of visit at POI \\
\texttt{avgDuration}  & (discretised) average duration at POI \\ \hline
\end{tabular}
\end{table}



\subsection{Evaluation metrics}
\label{sec:metric}

To evaluate the performance of a certain recommendation algorithm,
we need to measure the similarity (or loss) given prediction $\hat{\mathbf{y}}$ and ground truth $\mathbf{y}$.
Metrics researchers have used include
\begin{itemize}
\item Hamming loss $\frac{1}{K} \sum_{j=1}^K \llb \hat{y}_j \neq y_j \rrb$, this checks if every position is the same.

\item F$_1$ score on points~\cite{ijcai15}, where we care about the set of correctly recommended POIs. 
      Let $\texttt{set}(\mathbf{y})$ denote the set of POIs in trajectory $\mathbf{y}$, F$_1$ score on points is defined as
\begin{equation*}
F_1(\mathbf{y}, \hat{\mathbf{y}}) = \frac{2  P_{\textsc{point}}  R_{\textsc{point}}}{P_{\textsc{point}} + R_{\textsc{point}}}
%~~\text{where}~
%P_{\textsc{point}} = \frac{| \texttt{set}(\hat{\mathbf{y}}) \cap \texttt{set}(\mathbf{y}) |}{| \texttt{set}(\hat{\mathbf{y}}) |}~\text{and}~
%R_{\textsc{point}} = \frac{| \texttt{set}(\hat{\mathbf{y}}) \cap \texttt{set}(\mathbf{y}) |}{| \texttt{set}(\mathbf{y}) |}.
\end{equation*}
If $| \hat{\mathbf{y}} | = | \mathbf{y} |$, this metric is just the unordered Hamming loss, 
i.e., Hamming loss between two binary indicator vectors of size $| \mathcal{P} |$.

\item F$_1$ score on pairs~\cite{cikm16paper}, where we care about the set of correctly predicted POI pairs,
\begin{equation*}
\text{pairs-F}_1(\mathbf{y}, \hat{\mathbf{y}}) = \frac{2 P_{\textsc{pair}} R_{\textsc{pair}}}{P_{\textsc{pair}} + R_{\textsc{pair}}}
%~~\text{where}~
%P_{\textsc{pair}} = \frac{N_c} {| \texttt{set}(\hat{\mathbf{y}}) | (| \texttt{set}(\hat{\mathbf{y}}) | - 1) / 2}~\text{and}~
%R_{\textsc{pair}} = \frac{N_c} {| \texttt{set}(\mathbf{y}) | (| \texttt{set}(\mathbf{y}) | - 1) / 2},
\end{equation*}
and $N_c = \sum_{j=1}^{| \mathbf{y} | - 1} \sum_{k=j+1}^{| \mathbf{y} |} \llb y_j \prec_{\bar{\mathbf{y}}} y_k \rrb$,
here $y_j \prec_{\bar{\mathbf{y}}} y_k$ denotes that POI $y_j$ appears before POI $y_k$ in trajectory $\bar{\mathbf{y}}$.
We define pairs-F$_1 = 0$ when $N_c = 0$.

\end{itemize}

However, if we cast a trajectory $\mathbf{y} = (y_1,\dots,y_K)$ as a ranking of POIs in $\mathcal{P}$,
where $y_k$ has a rank $| \mathcal{P} | - k + 1$ and any other POI $p \notin \mathbf{y}$ has a rank $0$ ($0$ is an arbitrary choice).
We can make use of ranking evaluation metrics such as Kendall's $\tau$ by taking care of ties in ranks.

Given a prediction $\hat{\mathbf{y}} = (\hat{y}_1, \hat{y}_2, \dots, \hat{y}_K)$ and ground truth $\mathbf{y} = (y_1, y_2, \dots, y_K)$,
for a specific ordering of POIs $(p_1, p_2, \dots, p_{|\mathcal{P}|})$,
we produce two ranks according to $\mathbf{y}$ and $\hat{\mathbf{y}}$,
\begin{align*}
r_i       &= \sum_{j=1}^K (| \mathcal{P} | - j + 1)  \llb p_i = y_j \rrb,~
i = 1, \dots, | \mathcal{P} | \\
\hat{r}_i &= \sum_{j=1}^K (| \mathcal{P} | - j + 1)  \llb p_i = \hat{y}_j \rrb,~ 
i = 1, \dots, | \mathcal{P} |
\end{align*}
where POIs not in $\mathbf{y}$ will have a rank of $0$ in $r$ and similarly in $r$.
Then we have
\begin{align*}
C &= \frac{1}{2} \sum_{i,j} \left(\llb r_i < r_j \rrb  \llb \hat{r}_i < \hat{r}_j \rrb +
     \llb r_i > r_j \rrb  \llb \hat{r}_i > \hat{r}_j \rrb \right), \\
D &= \frac{1}{2} \sum_{i,j} \left(\llb r_i < r_j \rrb  \llb \hat{r}_i > \hat{r}_j \rrb +
     \llb r_i > r_j \rrb  \llb \hat{r}_i < \hat{r}_j \rrb \right), \\
T_{\mathbf{y}} &= \frac{1}{2} \sum_{i \ne j} \llb r_i = r_j \rrb \\
               &= \frac{1}{2} \sum_{i \ne j} \llb r_i = 0 \rrb  \llb r_j = 0 \rrb \\
               &= \frac{1}{2} \left( |\mathcal{P}| - K \right) \left( |\mathcal{P}| - K - 1 \right), \\ 
T_{\hat{\mathbf{y}}} &= \frac{1}{2} \sum_{i \ne j} \llb \hat{r}_i = \hat{r}_j \rrb \\
                     &= \frac{1}{2} \sum_{i \ne j} \llb \hat{r}_i = 0 \rrb  \llb \hat{r}_j = 0 \rrb \\
                     &= \frac{1}{2} \left( |\mathcal{P}| - K \right) \left( |\mathcal{P}| - K - 1 \right), \\ 
T_{\mathbf{y},\hat{\mathbf{y}}} &= \frac{1}{2} \sum_{i \ne j} \llb r_i = r_j \rrb  \llb \hat{r}_i = \hat{r}_j \rrb \\
                                &= \frac{1}{2} \sum_{i \ne j} \llb r_i = 0 \rrb  \llb r_j = 0 \rrb 
                                   \llb \hat{r}_i = 0 \rrb  \llb \hat{r}_j = 0 \rrb.
\end{align*}
Kendall's $\tau$ (version $b$)~\cite{kendall1945,agresti2010analysis} is
\begin{equation*}
\tau_b(\mathbf{y}, \hat{\mathbf{y}}) = \frac{C - D}{\sqrt{(C + D + T) (C + D + U)}},
\end{equation*}
where $T = T_{\mathbf{y}} - T_{\mathbf{y},\hat{\mathbf{y}}}$ and $U = T_{\hat{\mathbf{y}}} - T_{\mathbf{y},\hat{\mathbf{y}}}$.

Furthermore, F$_1$ score on points can be written as
\begin{equation*}
F_1(\mathbf{y}, \hat{\mathbf{y}}) = \frac{1}{K} \sum_i \llb r_i > 0 \rrb  \llb \hat{r}_i > 0 \rrb,
\end{equation*}
and F$_1$ score on pairs can be written as
\begin{align*}
& \text{pairs-F}_1(\mathbf{y}, \hat{\mathbf{y}}) \\
&= \left( \frac{1}{2} \sum_{i,j} \llb r_i < r_j \rrb  \llb r_i > 0 \rrb \llb \hat{r}_i < \hat{r}_j \rrb  \llb \hat{r}_i > 0 \rrb \right. \\
&  \left. ~+ \frac{1}{2} \sum_{i,j} \llb r_i > r_j \rrb  \llb r_j > 0 \rrb \llb \hat{r}_i > \hat{r}_j \rrb  \llb \hat{r}_j > 0 \rrb \right)
   \cdot \frac{1}{K(K-1)/2} \\
&= \frac{\sum_{i,j} \llb r_i < r_j \rrb  \llb r_i > 0 \rrb \llb \hat{r}_i < \hat{r}_j \rrb  \llb \hat{r}_i > 0 \rrb +
         \sum_{i,j} \llb r_i > r_j \rrb  \llb r_j > 0 \rrb \llb \hat{r}_i > \hat{r}_j \rrb  \llb \hat{r}_j > 0 \rrb} 
        {K(K-1)}
\end{align*}



\subsection{Evaluation Protocol}
\label{sec:protocol}

We first group trajectories according to queries that they conform to,
then evaluate the performance of each algorithm using nested cross validation,
where the hyper-parameters are optimised in the inner loop using Monte Carlo cross validation~\cite{burman1989comparative},
and the generalisation error is estimated in the outer loop using leave-one-out cross validation.

In particular, for query $\mathbf{x}^{(i)}$ and the set of trajectories $\mathcal{Y}^{(i)}$ that conform to $\mathbf{x}^{(i)}$,
the nested cross validation are performed over dataset $\left\{(\mathbf{x}^{(i)}, \mathcal{Y}^{(i)})\right\}_{i=1}^N$.

% metric: F_1, pairs F_1, Kendall's tau
In particular, given query $\mathbf{x}^{(i)}$ and the set of trajectories $\mathcal{Y}^{(i)}$ that conform to $\mathbf{x}^{(i)}$,
we measure the quality of the top-$k$ recommendations $\hat{\mathcal{Y}}^{(i)}$ 
by computing the metrics described in Section~\ref{sec:metric} as follows:
\begin{align*}
\text{F}_1 
&= \frac{1}{N} \sum_{i=1}^N \max_{\mathbf{y} \in \mathcal{Y}^{(i)}, \hat{\mathbf{y}} \in \hat{\mathcal{Y}}^{(i)}} \, 
   \text{F}_1(\mathbf{y}, \hat{\mathbf{y}}), \\
%\text{where F$_1'$ is the F$_1$ score on points for $\hat{\mathbf{y}}$ and $\mathbf{y}$} \\
\text{pairs-F}_1 
&= \frac{1}{N} \sum_{i=1}^N \max_{\mathbf{y} \in \mathcal{Y}^{(i)}, \hat{\mathbf{y}} \in \hat{\mathcal{Y}}^{(i)}} \, 
   \text{pairs-F}_1(\mathbf{y}, \hat{\mathbf{y}}), \\
%\text{where pairs-F$_1'$ is the F$_1$ score on pairs for $\hat{\mathbf{y}}$ and $\mathbf{y}$} \\
\tau 
&= \frac{1}{N} \sum_{i=1}^N \max_{\mathbf{y} \in \mathcal{Y}^{(i)}, \hat{\mathbf{y}} \in \hat{\mathcal{Y}}^{(i)}} \, 
   \tau(\mathbf{y}, \hat{\mathbf{y}}).
%\text{where $\tau'$ is the Kendall's $\tau$ for $\hat{\mathbf{y}}$ and $\mathbf{y}$}
\end{align*}



\subsection{Experimental Results}
\label{sec:result}



% Acknowledgements should only appear in the accepted version.
%\section*{Acknowledgements}
%
%\textbf{Do not} include acknowledgements in the initial version of
%the paper submitted for blind review.

\bibliographystyle{icml2017}
\bibliography{ref,ref_aditya}

\clearpage
\onecolumn

\section{Supplement}
\label{sec:supplement}

\subsection{Features}
\label{sec:feature}

The POI and query specific features extracted from trajectories are shown in Table~\ref{tab:poifeature},
features that describe the transition preference between different POIs are shown in Table~\ref{tab:tranfeature}.

\begin{table*}[ht]
\caption{Features of POI $p$ with respect to query $(s,K)$}
\label{tab:poifeature}
\centering


\setlength{\tabcolsep}{10pt} % tweak the space between columns
\begin{tabular}{l|l} \hline
\textbf{Feature}       & \textbf{Description} \\ \hline
\texttt{category}      & one-hot encoding of the category of $p$ \\
\texttt{neighbourhood} & one-hot encoding of the POI cluster that $p$ resides in \\
\texttt{popularity}    & logarithm of POI popularity of $p$ \\
\texttt{nVisit}        & logarithm of the total number of visit by all users at $p$ \\
\texttt{avgDuration}  & logarithm of the average visit duration at $p$ \\
\hline
%\texttt{nOccurrence}            & the number of times $p$ occurred in a trajectory that satisfies the query \\ DON'T know given new query

\texttt{trajLen}           & trajectory length $K$, i.e., the number of POIs required \\
\texttt{sameCatStart}      & $1$ if the category of $p$ is the same as that of $s$, $-1$ otherwise \\
\texttt{sameNeighbourhoodStart} & $1$ if $p$ resides in the same POI cluster as $s$, $-1$ otherwise \\
\texttt{diffPopStart}    & real-valued difference in POI popularity of $p$ from that of $s$ \\
\texttt{diffNVisitStart}        & real-valued difference in the total number of visit at $p$ from that at $s$ \\
\texttt{diffDurationStart}  & real-valued difference in average duration at $p$ from that at $s$ \\
\texttt{distStart}          & distance between $p$ and $s$, calculated using the Haversine formula \\
\hline
\end{tabular}
\end{table*}



\begin{table}[ht]
\caption{POI features used to estimate the (feature-wise) transition probabilities}
\label{tab:tranfeature}
\centering
\setlength{\tabcolsep}{2pt} % tweak the space between columns
\begin{tabular}{l|l} \hline
\textbf{Feature}       & \textbf{Description} \\ \hline
\texttt{category}      & category of POI \\
\texttt{neighbourhood} & the cluster that a POI resides in \\
\texttt{popularity}    & (discretised) popularity of POI \\
\texttt{nVisit}        & (discretised) total number of visit at POI \\
\texttt{avgDuration}  & (discretised) average duration at POI \\ \hline
\end{tabular}
\end{table}



\subsection{Evaluation metrics}
\label{sec:metric}

To evaluate the performance of a certain recommendation algorithm,
we need to measure the similarity (or loss) given prediction $\hat{\mathbf{y}}$ and ground truth $\mathbf{y}$.
Metrics researchers have used include
\begin{itemize}
\item Hamming loss $\frac{1}{K} \sum_{j=1}^K \llb \hat{y}_j \neq y_j \rrb$, this checks if every position is the same.

\item F$_1$ score on points~\cite{ijcai15}, where we care about the set of correctly recommended POIs. 
      Let $\texttt{set}(\mathbf{y})$ denote the set of POIs in trajectory $\mathbf{y}$, F$_1$ score on points is defined as
\begin{equation*}
F_1(\mathbf{y}, \hat{\mathbf{y}}) = \frac{2  P_{\textsc{point}}  R_{\textsc{point}}}{P_{\textsc{point}} + R_{\textsc{point}}}
%~~\text{where}~
%P_{\textsc{point}} = \frac{| \texttt{set}(\hat{\mathbf{y}}) \cap \texttt{set}(\mathbf{y}) |}{| \texttt{set}(\hat{\mathbf{y}}) |}~\text{and}~
%R_{\textsc{point}} = \frac{| \texttt{set}(\hat{\mathbf{y}}) \cap \texttt{set}(\mathbf{y}) |}{| \texttt{set}(\mathbf{y}) |}.
\end{equation*}
If $| \hat{\mathbf{y}} | = | \mathbf{y} |$, this metric is just the unordered Hamming loss, 
i.e., Hamming loss between two binary indicator vectors of size $| \mathcal{P} |$.

\item F$_1$ score on pairs~\cite{cikm16paper}, where we care about the set of correctly predicted POI pairs,
\begin{equation*}
\text{pairs-F}_1(\mathbf{y}, \hat{\mathbf{y}}) = \frac{2 P_{\textsc{pair}} R_{\textsc{pair}}}{P_{\textsc{pair}} + R_{\textsc{pair}}}
%~~\text{where}~
%P_{\textsc{pair}} = \frac{N_c} {| \texttt{set}(\hat{\mathbf{y}}) | (| \texttt{set}(\hat{\mathbf{y}}) | - 1) / 2}~\text{and}~
%R_{\textsc{pair}} = \frac{N_c} {| \texttt{set}(\mathbf{y}) | (| \texttt{set}(\mathbf{y}) | - 1) / 2},
\end{equation*}
and $N_c = \sum_{j=1}^{| \mathbf{y} | - 1} \sum_{k=j+1}^{| \mathbf{y} |} \llb y_j \prec_{\bar{\mathbf{y}}} y_k \rrb$,
here $y_j \prec_{\bar{\mathbf{y}}} y_k$ denotes that POI $y_j$ appears before POI $y_k$ in trajectory $\bar{\mathbf{y}}$.
We define pairs-F$_1 = 0$ when $N_c = 0$.

\end{itemize}

However, if we cast a trajectory $\mathbf{y} = (y_1,\dots,y_K)$ as a ranking of POIs in $\mathcal{P}$,
where $y_k$ has a rank $| \mathcal{P} | - k + 1$ and any other POI $p \notin \mathbf{y}$ has a rank $0$ ($0$ is an arbitrary choice).
We can make use of ranking evaluation metrics such as Kendall's $\tau$ by taking care of ties in ranks.

Given a prediction $\hat{\mathbf{y}} = (\hat{y}_1, \hat{y}_2, \dots, \hat{y}_K)$ and ground truth $\mathbf{y} = (y_1, y_2, \dots, y_K)$,
for a specific ordering of POIs $(p_1, p_2, \dots, p_{|\mathcal{P}|})$,
we produce two ranks according to $\mathbf{y}$ and $\hat{\mathbf{y}}$,
\begin{align*}
r_i       &= \sum_{j=1}^K (| \mathcal{P} | - j + 1)  \llb p_i = y_j \rrb,~
i = 1, \dots, | \mathcal{P} | \\
\hat{r}_i &= \sum_{j=1}^K (| \mathcal{P} | - j + 1)  \llb p_i = \hat{y}_j \rrb,~ 
i = 1, \dots, | \mathcal{P} |
\end{align*}
where POIs not in $\mathbf{y}$ will have a rank of $0$ in $r$ and similarly in $r$.
Then we have
\begin{align*}
C &= \frac{1}{2} \sum_{i,j} \left(\llb r_i < r_j \rrb  \llb \hat{r}_i < \hat{r}_j \rrb +
     \llb r_i > r_j \rrb  \llb \hat{r}_i > \hat{r}_j \rrb \right), \\
D &= \frac{1}{2} \sum_{i,j} \left(\llb r_i < r_j \rrb  \llb \hat{r}_i > \hat{r}_j \rrb +
     \llb r_i > r_j \rrb  \llb \hat{r}_i < \hat{r}_j \rrb \right), \\
T_{\mathbf{y}} &= \frac{1}{2} \sum_{i \ne j} \llb r_i = r_j \rrb \\
               &= \frac{1}{2} \sum_{i \ne j} \llb r_i = 0 \rrb  \llb r_j = 0 \rrb \\
               &= \frac{1}{2} \left( |\mathcal{P}| - K \right) \left( |\mathcal{P}| - K - 1 \right), \\ 
T_{\hat{\mathbf{y}}} &= \frac{1}{2} \sum_{i \ne j} \llb \hat{r}_i = \hat{r}_j \rrb \\
                     &= \frac{1}{2} \sum_{i \ne j} \llb \hat{r}_i = 0 \rrb  \llb \hat{r}_j = 0 \rrb \\
                     &= \frac{1}{2} \left( |\mathcal{P}| - K \right) \left( |\mathcal{P}| - K - 1 \right), \\ 
T_{\mathbf{y},\hat{\mathbf{y}}} &= \frac{1}{2} \sum_{i \ne j} \llb r_i = r_j \rrb  \llb \hat{r}_i = \hat{r}_j \rrb \\
                                &= \frac{1}{2} \sum_{i \ne j} \llb r_i = 0 \rrb  \llb r_j = 0 \rrb 
                                   \llb \hat{r}_i = 0 \rrb  \llb \hat{r}_j = 0 \rrb.
\end{align*}
Kendall's $\tau$ (version $b$)~\cite{kendall1945,agresti2010analysis} is
\begin{equation*}
\tau_b(\mathbf{y}, \hat{\mathbf{y}}) = \frac{C - D}{\sqrt{(C + D + T) (C + D + U)}},
\end{equation*}
where $T = T_{\mathbf{y}} - T_{\mathbf{y},\hat{\mathbf{y}}}$ and $U = T_{\hat{\mathbf{y}}} - T_{\mathbf{y},\hat{\mathbf{y}}}$.

Furthermore, F$_1$ score on points can be written as
\begin{equation*}
F_1(\mathbf{y}, \hat{\mathbf{y}}) = \frac{1}{K} \sum_i \llb r_i > 0 \rrb  \llb \hat{r}_i > 0 \rrb,
\end{equation*}
and F$_1$ score on pairs can be written as
\begin{align*}
& \text{pairs-F}_1(\mathbf{y}, \hat{\mathbf{y}}) \\
&= \left( \frac{1}{2} \sum_{i,j} \llb r_i < r_j \rrb  \llb r_i > 0 \rrb \llb \hat{r}_i < \hat{r}_j \rrb  \llb \hat{r}_i > 0 \rrb \right. \\
&  \left. ~+ \frac{1}{2} \sum_{i,j} \llb r_i > r_j \rrb  \llb r_j > 0 \rrb \llb \hat{r}_i > \hat{r}_j \rrb  \llb \hat{r}_j > 0 \rrb \right)
   \cdot \frac{1}{K(K-1)/2} \\
&= \frac{\sum_{i,j} \llb r_i < r_j \rrb  \llb r_i > 0 \rrb \llb \hat{r}_i < \hat{r}_j \rrb  \llb \hat{r}_i > 0 \rrb +
         \sum_{i,j} \llb r_i > r_j \rrb  \llb r_j > 0 \rrb \llb \hat{r}_i > \hat{r}_j \rrb  \llb \hat{r}_j > 0 \rrb} 
        {K(K-1)}
\end{align*}


\begin{algorithm}[htbp]
\caption{The list Viterbi algorithm for inference}
\label{alg:listviterbi}
\begin{algorithmic}[1]
\STATE \textbf{Input}: $\mathbf{x}=(s, K),~ \mathcal{P},~ \mathbf{w},~ \Psi,~ l$
%\STATE Initialise score matrices $\alpha,~ \beta,~ f_t,~ f_{t, t+1}$, a max-heap $H,~ k=0$.
\STATE Initialise score matrices $\alpha,~ \beta,~ f_{t, t+1}$, a max-heap $H$, result set $R$, $k=0$.
\STATE $\triangleright$ Do the forward-backward procedure~\cite{rabiner1989tutorial}
\STATE $\forall p_j \in \mathcal{P},~ \alpha_t(p_j) = 
        \begin{cases}
        0,~ t = 1 \\
        \max_{p_i \in \mathcal{P}} \left\{ \alpha_{t-1}(p_i) + \mathbf{w}_{ij}^\top \Psi_{ij}(\mathbf{x}, p_i, p_j) + 
        \mathbf{w}_j^\top \Psi_j(\mathbf{x}, p_j) \right\},~ t=2,\dots,K
        \end{cases}$

\STATE $\forall p_i \in \mathcal{P},~ \beta_t(p_i) = 
        \begin{cases}
        0,~ t = K \\
        \max_{p_j \in \mathcal{P}} \left\{ \mathbf{w}_{ij}^\top \Psi_{ij}(\mathbf{x}, p_i, p_j) + 
        \mathbf{w}_j^\top \Psi_j(\mathbf{x}, p_j) + \beta_{t+1}(p_j) \right\},~ t = K-1,\dots,1
        \end{cases}$

%\STATE $\forall p_i \in \mathcal{P},~ f_t(p_i) = \alpha_t(p_i) + \beta_t(p_i),~ t = 1,\dots,K$
\STATE $\forall p_i, p_j \in \mathcal{P},~ f_{t,t+1}(p_i, p_j) = \alpha_t(p_i) + \mathbf{w}_{ij}^\top \Psi_{ij}(\mathbf{x}, p_i, p_j) + 
                              \mathbf{w}_j^\top \Psi_j(\mathbf{x}, p_j) + \beta_{t+1}(p_j),~ t = 1,\dots,K-1$

\STATE $\triangleright$ Identify the best (scored) trajectory $\mathbf{y}^1=(y_1^1,\dots,y_K^1)$ (possibly with sub-tours)
\STATE $y_t^1 = \begin{cases}
                s,~ t = 1 \\
%                \argmax_{p \in \mathcal{P}} \left\{ f_{1,2}(s, p) \right\},~ t = 2, \\
                \argmax_{p \in \mathcal{P}} \left\{ f_{t-1,t}(y_{t-1}^1, p) \right\},~ t = 2,\dots,K
                \end{cases}$

%\STATE $r^1 = \max_{p \in \mathcal{P}} \left\{ f_K(p) \right\}~~~ \triangleright$ $r^1$ is the score/priority of $\mathbf{y}^1$
\STATE $r^1 = \max_{p \in \mathcal{P}} \left\{ \alpha_{K}(p) \right\}~~~ \triangleright$ $r^1$ is the score/priority of $\mathbf{y}^1$
\STATE $H.\textit{push}\left(r^1,~ (\mathbf{y}^1, \textsc{nil}, \emptyset) \right)$

\WHILE{$H \ne \emptyset$ \textbf{and} $k < \,|\mathcal{P}|^{K-1} - \prod_{t=2}^K (|\mathcal{P}|-t+1)$}
    \STATE $r^k,~ (\mathbf{y}^k, I, S) = H.\textit{pop}()~~~ \triangleright$ 
           $r^k$ is the score of $\mathbf{y}^k=(y_1^k,\dots,y_K^k)$, $I$ is the partition index, and $S$ is the exclude set
    \STATE $k = k + 1$
    \STATE Add $\mathbf{y}^k$ to $R$ if NO sub-tours in $\mathbf{y}^k$
    \RETURN $R$ if it contains $l$ trajectories
    \STATE $\bar{I} = \begin{cases}
                      2,~ I = \textsc{nil} \\
                      I,~ \text{otherwise}
                      \end{cases}$

    \FOR{$t = \bar{I},\dots,K$}
        \STATE $\bar{S} = \begin{cases}
                          S \cup \{ y_t^k \},~ t = \bar{I} \\
                          \{ y_t^k \},~ \text{otherwise}
                          \end{cases}$

        \STATE $\bar{y}_j = \begin{cases}
                            y_j^k,~~ j=1,\dots,t-1 \\
                            %\argmax_{p \in \mathcal{P} \setminus \textit{new\_exclude\_set}} f_{t-1,t}(y_{t-1}^k, p),~ j=t \\
                            \argmax_{p \in \mathcal{P} \setminus \bar{S}} \left\{ f_{t-1,t}(y_{t-1}^k, p) \right\},~ j=t \\
                            \argmax_{p \in \mathcal{P}} \left\{ f_{j-1, j}(\bar{y}_{j-1}, p) \right\},~ j=t+1,\dots,K
                \end{cases}$
        \STATE $\bar{r} = \begin{cases}
                          f_{t-1,t}(y_{t-1}^k, \bar{y}_t),~ I = \textsc{nil} \\
                          r^k + f_{t-1,t}(y_{t-1}^k, \bar{y}_t) - f_{t-1,t}(y_{t-1}^k, y_t^k),~ \text{otherwise}
                          \end{cases}$

        $H.\textit{push}\left(\bar{r}, (\bar{\mathbf{y}}, t, \bar{S}) \right)$
    \ENDFOR
\ENDWHILE
\end{algorithmic}
\end{algorithm}


\end{document}
