%!TEX root = main.tex

% sub-tour elimination
\subsection{Sequence decoding for the SR model}
\label{ssec:subtour}

The SR model require two algorithmic components for inference and learning that is missing from the SSVM algorithms. 

For inference, SR needs to predict a {\em path}, \ie a sequence whose elements are distinct from each other. 
As described in Section~\ref{sec:intro}, this is desirable since users traversing a sequence (of locations or music)
would not want to see repeated entries. 
Given desired sequence length $l$ among $m$ possible points, the Viterbi~\cite{tsochantaridis2005large} 
will generate the length-$l$ sequence with the best score, i.e. $\y^* = \argmax_\y f(\x,\y)$. 
One may then use the well-known 
Christofes algorithm~\cite{christofides1976} on $y^*$ to eliminate loops in the sequence. 
%is known to generate a solutions within a factor of 3/2 of the optimal solution for traveling salesman problems. 
However, the resulting sequence will have less than the desired number of points, and the resulting score won't be optimal. 

For learning an SR model, loss-augmented inference (\ref{eq:loss_aug_inf}) needs to be done by excluding multiple known sequences. 
As described in Section~\ref{ssec:sr}, %this involves %we would want to maximize the loss-augmented objective function 
%$\max_{\bar\y} \left\{ \w^\top \Psi(\x^{(i)}, \bar\y) + \Delta(\y^{(ij)}, \bar{\y}) \right\}$
%where the domain of candidate sequences excludes the known sequences for query $\x^{(i)}$, \ie $\bar{\y} \in \mathcal{Y} \setminus \{\y^{(ij)}\}_{j=1}^{n_i}$. 
however,the Viterbi algorithm use back-tracking to find the best sequence, 
and cannot easily exclude known sequences. 

%Both the loss-augmented inference and prediction inference for structured SVM (Section~\ref{sec:ssvm}) cannot be done efficiently 
%if the no duplicates constraints are required, moreover, we would like to recommend more than one trajectories given a query.

The rest of this section describes two algorithms, each intuitively aimed to address one of the two requirements above. 
Both can be applied in novel contexts of the SR model. 
We will also describe practical choices about which algorithm to use when. 


\subsubsection{Traversing points with integer linear programming}
% ILP for subtour elimination
Inference in the SR model requires finding the best path that traverses exactly $l$ of $m$ candidate points. 
This is can be seen as a variant of the traveling salesman problem (TSP), or the best of ${m \choose l}$ TSPs.
Such a point traversal problem can be solved by incorporating 
sub-tour elimination constraints of the travelling salesman problem (TSP).
In particular, the following integer linear program (ILP) formulation~\cite{ijcai15,cikm16paper} 
will solve SR inference for trajectory $\y$ of length $l$. 
\resizebox{.99\columnwidth}{!}{
  \begin{minipage}{\linewidth}
\begin{alignat}{5}
\max_{\bu,\bv} ~& \sum_{k=1}^m \w_k^\top \Psi_k(\x, y_k) \sum_{j=1}^m u_{jk} 
+ \sum_{j,k=1}^m u_{jk} \w_{jk}^\top \Psi_{j, k}(\x, y_j, y_k) \nonumber\\
  %              & + \sum_{j=1}^l \sum_{k=1}^l u_{jk} \w_{jk}^\top \Psi_{j, k}(\x, y_j, y_k) \\
s.t. ~& u_{jk}, ~z_j \in \{0, 1\}, ~u_{jj}=0, ~z_1=0, ~v_j \in \mathbf{Z},~                          \label{eq:cons1} \\
  & y_j \in \{1,\dots,m\}, ~\forall j, k = 1,\cdots,m                                                \nonumber \\
  & \sum_{k=2}^m u_{1k} = 1, ~\sum_{j=2}^m u_{j1} = 0  \label{eq:cons2} \\
  & \sum_{j=1}^m u_{ji} = z_i + \sum_{k=2}^m u_{ik} \le 1,  ~\forall i=2,\cdots,m                    \label{eq:cons3} \\
  & \sum_{j=1}^m \sum_{k=1}^m u_{jk} = l-1,                                                          \label{eq:cons4} \\
  & v_j - v_k + 1 \le (m-1) (1-u_{jk}),                     ~\forall j,k=2,\cdots,m                  \nonumber
\end{alignat}
\end{minipage}
}
Here $u_{jk}$ are binary decision variables that is true iff the transition from $y_j$ to $y_k$ is in the resulting trajectory,
$z_j$ is a binary decision variable that is true iff $y_j$ is the last POI in trajectory.
$l$ is the number of candidate POIs.
For brevity, we index the POIs such that $y_1 = 1$.
Firstly, the desired trajectory should start from $y_1$ (Constraint~\ref{eq:cons2}).
In addition, any POI could be visited at most once (Constraint~\ref{eq:cons3}).
Moreover, only $l-1$ transitions between POIs are permitted 
since the trajectory length is $l$ (Constraint~\ref{eq:cons4}).
The last constraint, where $v_j$ is an auxiliary variable,
enforces that only a single sequence of POIs without sub-tours is permitted in the trajectory.
%\TODO{LX: i do not understand the last contraint, $v_j$ did not seem to have been defined? are they binary or something else?} 
% DW: v_j defined in the first constraint, which is a natural number

TSPs are NP-hard, the ILP formulation will give the optimal solution when the solver finds the optimal. 
We note, however, that and ILP cannot be readily used for loss-augmented inference
for the Hamming loss, the most common loss function for sequence prediction tasks. 
This is because ILP requires the loss should be a linear function of $u_{jk}$,
\eg, the number of mis-predicted POIs disregarding the order $\Delta(\y, \bar\y) = 1 - \sum_{j=1}^m \sum_{k=1}^m u_{j, y_k}$. 
But Hamming loss cannot be expressed as a linear function of $u_{jk}$.
%if we define the loss as the number of mispredicted POIs,
%where $\mathbf{y}$ is the ground truth and $\bar{\mathbf{y}}$ is the trajectory corresponding to the optimal solution of this ILP.
%=======
%If we employ the above ILP to do loss-augmented inference, one restriction is that the loss should be a linear function of $u_{jk}$,
%e.g., $\Delta(\y, \bar{\y}) = 1 - \sum_{j=1}^M \sum_{k=1}^M u_{j, y_k}$ if we define the loss as the number of mispredicted POIs,
%where $\y$ is the ground truth and $\bar{\y}$ is the trajectory corresponding to the optimal solution of this ILP.

\subsubsection{Finding $k$-best sequences}
% 2 uses of list Viterbi: 1) multiple ground truths; 2) subtour elimination

The algorithm that seem closest to loss-augmented inference with exclusions are several extensions 
of the Viterbi algorithm that aims to find more than one high-scoring sequences. 
The parallel listViterbi algorithm~\cite{seshadri1994list} finds the top $k$ sequences 
by keeping $k$ backtrack pointers for each possible state in each position of the sequence. 
This algorithm is computationally efficient 
-- a factor of $\log L$ more than the standard Viterbi for score sorting in each step, rather than simple maximization 
-- but it keeps many unnecessary pointers. It is also difficult to apply to the SR inference scenario, 
since we do not know $k$ beforehand. This is because it is generally impossible to foresee 
the rank (according to score) of the first path among all valid sequences (that may have repeats). 

We resort to the serial list Viterbi algorithm~\cite{seshadri1994list,nilsson2001sequentially}. 
These algorithms sequentially find the $k$-th best sequence given the best, $2$nd best, \dots, $(k-1)$-th best sequences. 
The key insight here is that the 2nd best sequence deviates from the best sequence
for one continuous segment, and then finally merges back to the best sequence without deviating again 
-- otherwise replacing one of the continuous segments with those from the best sequence will increase the score. 
Subsequently, the $k$th best sequence can be the 2nd best sequence relative to the $k-1$th sequence 
at the point of final merge, or the 2nd or 3rd best sequence to the $k-1$th sequence at the point of final merge, \ldots, and so on. 
%The version of serial list Viterbi presented by~\cite{nilsson2001sequentially} accommodates 
The serial list Viterbi allows exclusion of sequences with repeats, by checking whether or not the current $k$th best sequence has a repeat, if it does, discard and proceed to the $k+1$th. It also allows excluding an arbitrary number of known sequences, by additionally checking, when we get each of the $k$th best sequence, whether or not it's in the exclusion set. 
%by partitioning the remaining space 
%from each of the known sequence according to their points of deviation, 
%finding the best trajectory within each partition and then identifying the best among different partitions. 
Due to space limitations, the list Viterbi algorithm is fully detailed in the appendix\footnote{add link to appendix? \TODO{dawei?}}. 

For the structured recommendation problem, the serial list Viterbi algorithm can be used for inference 
by recursively getting the next best sequence, until one or more paths are found. 
This algorithm can also be used for loss-augmented inference with Hamming loss, 
since it only require the loss function be decomposable with respect to the label elements. 
%This list Viterbi algorithm performs a backward search for trajectories that merges into the existing candidates at any given point. 
%as described in Algorithm~\ref{alg:listviterbi}.

%\TODO{Briefly discuss which one to use, when}
