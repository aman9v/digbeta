\section{Experiment}
\label{sec:experiment}

\subsection{Dataset and Features}
\label{sec:dataset}

% experiment protocol: Nested cross-validation with Monte-Carlo cross-validation for the inner loop
We experiment methods developed in Section~\ref{sec:method} on trajectories extracted from Flickr photos~\cite{thomee2016yfcc100m}.

\subsection{Features}
\label{sec:feature}

The POI and query specific features extracted from trajectories are shown in Table~\ref{tab:poifeature},
features that describe the transition preference between different POIs are shown in Table~\ref{tab:tranfeature}.

\begin{table*}[ht]
\caption{Features of POI $p$ with respect to query $(s,K)$}
\label{tab:poifeature}
\centering


\setlength{\tabcolsep}{10pt} % tweak the space between columns
\begin{tabular}{l|l} \hline
\textbf{Feature}       & \textbf{Description} \\ \hline
\texttt{category}      & one-hot encoding of the category of $p$ \\
\texttt{neighbourhood} & one-hot encoding of the POI cluster that $p$ resides in \\
\texttt{popularity}    & logarithm of POI popularity of $p$ \\
\texttt{nVisit}        & logarithm of the total number of visit by all users at $p$ \\
\texttt{avgDuration}  & logarithm of the average visit duration at $p$ \\
\hline
%\texttt{nOccurrence}            & the number of times $p$ occurred in a trajectory that satisfies the query \\ DON'T know given new query

\texttt{trajLen}           & trajectory length $K$, i.e., the number of POIs required \\
\texttt{sameCatStart}      & $1$ if the category of $p$ is the same as that of $s$, $-1$ otherwise \\
\texttt{sameNeighbourhoodStart} & $1$ if $p$ resides in the same POI cluster as $s$, $-1$ otherwise \\
\texttt{diffPopStart}    & real-valued difference in POI popularity of $p$ from that of $s$ \\
\texttt{diffNVisitStart}        & real-valued difference in the total number of visit at $p$ from that at $s$ \\
\texttt{diffDurationStart}  & real-valued difference in average duration at $p$ from that at $s$ \\
\texttt{distStart}          & distance between $p$ and $s$, calculated using the Haversine formula \\
\hline
\end{tabular}
\end{table*}



\begin{table}[ht]
\caption{POI features used to estimate the (feature-wise) transition probabilities}
\label{tab:tranfeature}
\centering
%\setlength{\tabcolsep}{28pt} % tweak the space between columns
\begin{tabular}{l|l} \hline
\textbf{Feature}       & \textbf{Description} \\ \hline
\texttt{category}      & category of POI \\
\texttt{neighbourhood} & the cluster that a POI resides in \\
\texttt{popularity}    & (discretised) popularity of POI \\
\texttt{nVisit}        & (discretised) total number of visit at POI \\
\texttt{avgDuration}  & (discretised) average duration at POI \\ \hline
\end{tabular}
\end{table}



\subsection{Evaluation metrics}
\label{sec:metric}

To evaluate the performance of a certain recommendation algorithm,
we need to measure the similarity (or loss) given prediction $\hat{\mathbf{y}}$ and ground truth $\mathbf{y}$.
Metrics researchers have used include
\begin{itemize}
\item Hamming loss $\frac{1}{K} \sum_{j=1}^K \llb \hat{y}_j \neq y_j \rrb$, this checks if every position is the same.

\item F$_1$ score on points~\cite{ijcai15}, where we care about the set of correctly recommended POIs. 
      Let $\texttt{set}(\mathbf{y})$ denote the set of POIs in trajectory $\mathbf{y}$, F$_1$ score on points is defined as
\begin{equation*}
F_1(\mathbf{y}, \hat{\mathbf{y}}) = \frac{2  P_{\textsc{point}}  R_{\textsc{point}}}{P_{\textsc{point}} + R_{\textsc{point}}} ~~\text{where}~
P_{\textsc{point}} = \frac{| \texttt{set}(\hat{\mathbf{y}}) \cap \texttt{set}(\mathbf{y}) |}{| \texttt{set}(\hat{\mathbf{y}}) |}~\text{and}~
R_{\textsc{point}} = \frac{| \texttt{set}(\hat{\mathbf{y}}) \cap \texttt{set}(\mathbf{y}) |}{| \texttt{set}(\mathbf{y}) |}.
\end{equation*}
If $| \hat{\mathbf{y}} | = | \mathbf{y} |$, this metric is just the unordered Hamming loss, 
i.e., Hamming loss between two binary indicator vectors of size $| \mathcal{P} |$.

\item F$_1$ score on pairs~\cite{cikm16paper}, where we care about the set of correctly predicted POI pairs,
\begin{equation*}
\text{pairs-F}_1(\mathbf{y}, \hat{\mathbf{y}}) = \frac{2 P_{\textsc{pair}} R_{\textsc{pair}}}{P_{\textsc{pair}} + R_{\textsc{pair}}}~~\text{where}~
P_{\textsc{pair}} = \frac{N_c} {| \texttt{set}(\hat{\mathbf{y}}) | (| \texttt{set}(\hat{\mathbf{y}}) | - 1) / 2}~\text{and}~
R_{\textsc{pair}} = \frac{N_c} {| \texttt{set}(\mathbf{y}) | (| \texttt{set}(\mathbf{y}) | - 1) / 2},
\end{equation*}
and $N_c = \sum_{j=1}^{| \mathbf{y} | - 1} \sum_{k=j+1}^{| \mathbf{y} |} \llb y_j \prec_{\bar{\mathbf{y}}} y_k \rrb$,
here $y_j \prec_{\bar{\mathbf{y}}} y_k$ denotes that POI $y_j$ appears before POI $y_k$ in trajectory $\bar{\mathbf{y}}$.
We define pairs-F$_1 = 0$ when $N_c = 0$.

\end{itemize}

However, if we cast a trajectory $\mathbf{y} = (y_1,\dots,y_K)$ as a ranking of POIs in $\mathcal{P}$,
where $y_k$ has a rank $| \mathcal{P} | - k + 1$ and any other POI $p \notin \mathbf{y}$ has a rank $0$ ($0$ is an arbitrary choice).
We can make use of ranking evaluation metrics such as Kendall's $\tau$ by taking care of ties in ranks.

Given a prediction $\hat{\mathbf{y}} = (\hat{y}_1, \hat{y}_2, \dots, \hat{y}_K)$ and ground truth $\mathbf{y} = (y_1, y_2, \dots, y_K)$,
for a specific ordering of POIs $(p_1, p_2, \dots, p_{|\mathcal{P}|})$,
we produce two ranks according to $\mathbf{y}$ and $\hat{\mathbf{y}}$,
\begin{align*}
r_i       &= \sum_{j=1}^K (| \mathcal{P} | - j + 1)  \llb p_i = y_j \rrb,~
i = 1, \dots, | \mathcal{P} | \\
\hat{r}_i &= \sum_{j=1}^K (| \mathcal{P} | - j + 1)  \llb p_i = \hat{y}_j \rrb,~ 
i = 1, \dots, | \mathcal{P} |
\end{align*}
where POIs not in $\mathbf{y}$ will have a rank of $0$ in $r$ and similarly in $r$.
Then we have
\begin{align*}
C &= \frac{1}{2} \sum_{i,j} \left(\llb r_i < r_j \rrb  \llb \hat{r}_i < \hat{r}_j \rrb +
     \llb r_i > r_j \rrb  \llb \hat{r}_i > \hat{r}_j \rrb \right), \\
D &= \frac{1}{2} \sum_{i,j} \left(\llb r_i < r_j \rrb  \llb \hat{r}_i > \hat{r}_j \rrb +
     \llb r_i > r_j \rrb  \llb \hat{r}_i < \hat{r}_j \rrb \right), \\
T_{\mathbf{y}} &= \frac{1}{2} \sum_{i \ne j} \llb r_i = r_j \rrb 
                = \frac{1}{2} \sum_{i \ne j} \llb r_i = 0 \rrb  \llb r_j = 0 \rrb 
                = \frac{1}{2} \left( |\mathcal{P}| - K \right) \left( |\mathcal{P}| - K - 1 \right), \\ 
T_{\hat{\mathbf{y}}} &= \frac{1}{2} \sum_{i \ne j} \llb \hat{r}_i = \hat{r}_j \rrb
                      = \frac{1}{2} \sum_{i \ne j} \llb \hat{r}_i = 0 \rrb  \llb \hat{r}_j = 0 \rrb
                      = \frac{1}{2} \left( |\mathcal{P}| - K \right) \left( |\mathcal{P}| - K - 1 \right), \\ 
T_{\mathbf{y},\hat{\mathbf{y}}} &= \frac{1}{2} \sum_{i \ne j} \llb r_i = r_j \rrb  \llb \hat{r}_i = \hat{r}_j \rrb
                                 = \frac{1}{2} \sum_{i \ne j} \llb r_i = 0 \rrb  \llb r_j = 0 \rrb 
                                   \llb \hat{r}_i = 0 \rrb  \llb \hat{r}_j = 0 \rrb
                                 = \frac{1}{2} {d(d-1)},
\end{align*}
Kendall's $\tau$ (version $b$)~\cite{kendall1945,agresti2010analysis} is
\begin{equation*}
\tau_b(\mathbf{y}, \hat{\mathbf{y}}) = \frac{C - D}{\sqrt{(C + D + T) (C + D + U)}},
\end{equation*}
where $T = T_{\mathbf{y}} - T_{\mathbf{y},\hat{\mathbf{y}}}$ and $U = T_{\hat{\mathbf{y}}} - T_{\mathbf{y},\hat{\mathbf{y}}}$.

Furthermore, F$_1$ score on points can be written as
\begin{equation*}
F_1(\mathbf{y}, \hat{\mathbf{y}}) = \frac{1}{K} \sum_i \llb r_i > 0 \rrb  \llb \hat{r}_i > 0 \rrb,
\end{equation*}
and F$_1$ score on pairs can be written as
\begin{equation*}
\text{pairs-F}_1(\mathbf{y}, \hat{\mathbf{y}}) 
= \frac{\frac{1}{2} \sum_{i,j} 
  \llb r_i < r_j \rrb  \llb r_i > 0 \rrb  
  \llb \hat{r}_i < \hat{r}_j \rrb  \llb \hat{r}_i > 0 \rrb + 
  \frac{1}{2} \sum_{i,j} 
  \llb r_i > r_j \rrb  \llb r_j > 0 \rrb 
  \llb \hat{r}_i > \hat{r}_j \rrb  \llb \hat{r}_j > 0 \rrb}{K(K-1)/2}.
\end{equation*}



\subsection{Evaluation Protocol}
\label{sec:protocol}

We first group trajectories according to queries that they conform to,
then evaluate the performance of each algorithm using nested cross validation,
where the hyper-parameters are optimised in the inner loop using Monte Carlo cross validation~\cite{burman1989comparative},
and the generalisation error is estimated in the outer loop using leave-one-out cross validation.

In particular, for query $\mathbf{x}^{(i)}$ and the set of trajectories $\mathcal{Y}^{(i)}$ that conform to $\mathbf{x}^{(i)}$,
the nested cross validation are performed over dataset $\left\{(\mathbf{x}^{(i)}, \mathcal{Y}^{(i)})\right\}_{i=1}^N$.

% metric: F_1, pairs F_1, Kendall's tau
In particular, given query $\mathbf{x}^{(i)}$ and the set of trajectories $\mathcal{Y}^{(i)}$ that conform to $\mathbf{x}^{(i)}$,
we measure the quality of the top-$k$ recommendations $\hat{\mathcal{Y}}^{(i)}$ 
by computing the metrics described in Section~\ref{sec:metric} as follows:
\begin{align*}
\text{F}_1 
&= \frac{1}{N} \sum_{i=1}^N \max_{\mathbf{y} \in \mathcal{Y}^{(i)}, \hat{\mathbf{y}} \in \hat{\mathcal{Y}}^{(i)}} \, 
   \text{F}_1(\mathbf{y}, \hat{\mathbf{y}}), \\
%\text{where F$_1'$ is the F$_1$ score on points for $\hat{\mathbf{y}}$ and $\mathbf{y}$} \\
\text{pairs-F}_1 
&= \frac{1}{N} \sum_{i=1}^N \max_{\mathbf{y} \in \mathcal{Y}^{(i)}, \hat{\mathbf{y}} \in \hat{\mathcal{Y}}^{(i)}} \, 
   \text{pairs-F}_1(\mathbf{y}, \hat{\mathbf{y}}), \\
%\text{where pairs-F$_1'$ is the F$_1$ score on pairs for $\hat{\mathbf{y}}$ and $\mathbf{y}$} \\
\tau 
&= \frac{1}{N} \sum_{i=1}^N \max_{\mathbf{y} \in \mathcal{Y}^{(i)}, \hat{\mathbf{y}} \in \hat{\mathcal{Y}}^{(i)}} \, 
   \tau(\mathbf{y}, \hat{\mathbf{y}}).
%\text{where $\tau'$ is the Kendall's $\tau$ for $\hat{\mathbf{y}}$ and $\mathbf{y}$}
\end{align*}



\subsection{Experimental Results}
\label{sec:result}
