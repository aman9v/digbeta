%\documentclass[10pt,a4paper]{article}
%\documentclass[twocolumn,10pt,a4paper]{article}
%\documentclass[twocolumn,a4wide,9pt]{extarticle}
\documentclass[9pt]{extarticle}
\usepackage[a4paper,top=0.85in,left=0.75in,bottom=1in,right=0.52in]{geometry} % A4 paper margins
%\usepackage[a4paper,top=0.75in,left=0.1in,bottom=0.9in,right=0.1in]{geometry} % A4 paper margins
\usepackage{hyperref}
\usepackage{amsmath}
\usepackage{amsfonts}
\usepackage{mathrsfs}
\usepackage{bm}
\usepackage{bbm}
\usepackage{algorithm}
\usepackage{algorithmic}
\usepackage[sc]{mathpazo}
\linespread{1.05}         % Palladio needs more leading (space between lines)
\usepackage[T1]{fontenc}

\DeclareMathOperator*{\argmin}{argmin}
\DeclareMathOperator*{\argmax}{argmax}
\newcommand{\eat}[1]{}
\setlength{\columnsep}{1.5em} % spacing between columns

\title{The Trajectory Recommendation Problem}

\author{Dawei Chen}

\date{\today}

\begin{document}

\maketitle


\section{Problem formulation}
\label{sec:formulation}

Given a set of points of interest (POI) $\mathcal{P}$ and a trajectory query $\mathbf{x} = (s, K)$,
where $s \in \mathcal{P}$ is the desired start POI and $K$ is the number of POIs in the desired trajectory (including the start location $s$).
We want to recommend a sequence of POIs $\mathbf{y}^*$ that maximises utility, i.e., for a suitable function $f(\cdot,\cdot)$,
\begin{equation*}
\mathbf{y}^* = \argmax_{\mathbf{y}}~f(\mathbf{x}, \mathbf{y}),
\end{equation*}
where $\mathbf{y} = (y_1 = s,~ y_2, \dots, y_K)$ is a trajectory with $K$ POIs, $y_k \in \mathcal{P},~ k=1,\dots,K$, and $y_j \ne y_k$ if $j \ne k$.

On the other hand, we can constrain the desired trajectory with a total time budget $T$ instead of the number of POIs.
In this case, the number of POIs $K$ can be treated as a \emph{hidden} variable, with additional constraint $\sum_{k=1}^K t_k \le T$ 
where $t_k$ is the time spent at POI $y_k$.



\subsection{Related problems}
\label{sec:related}

This problem is related to automatic playlist generation, 
where we recommend a sequence of songs given a specified song (a.k.a. the seed) and the number of new songs.
Formally, given a library of songs and a query $\mathbf{x} = (s, K)$, where $s$ is the seed and $K$ is the number of songs in playlist,
we produce a list with $K$ songs (without duplication) by maximising the likelihood~\cite{chen2012playlist},
\begin{equation*}
%\max_{(y_1,\dots,y_K)} \prod_{k=2}^K \mathbb{P}(y_{k-1} \mid y_k),~ y_1 = s ~\text{and}~ y_j \ne y_k,~ j \ne k.
\mathbf{y}^* = \argmax_{\mathbf{y}} \mathbb{P}(\mathbf{y} \mid \mathbf{x}),~ \mathbf{y} = (y_1=s,\dots,y_K) ~\text{and}~ y_j \ne y_k ~\text{if}~ j \ne k.
\end{equation*}

Another similar problem is choosing a small set of photos from a large photo library and compiling them into a slideshow or movie.



\section{Proposed methods}
\label{sec:methods}

In general, there are two approaches to solve this problem,
\begin{enumerate}
\item \emph{Scoring POIs} (independently), and then pick the $K-1$ highest scored POIs from $\mathcal{P} \setminus s$,
\item \emph{Scoring trajectories}, and then pick the highest scored trajectory with respect to query $\mathbf{x}$.
\end{enumerate}

In this section, we briefly describe a variety of methods that following these two approaches.
Suppose the training set contains $N$ trajectories 
$\{ \mathbf{x}^{(i)}, \mathbf{y}^{(i)} \}_{i=1}^N$,
where $\mathbf{y}^{(i)}$ is the $i$-th trajectory and $\mathbf{x}^{(i)} = (y_1,~ \mid \mathbf{y}^{(i)} \mid)$ is the query 
with respect to trajectory $\mathbf{y}^{(i)}$.



\subsection{POI scoring}
\label{sec:scoring_point}

We have a POI scoring function $S: \mathcal{X} \to \mathbb{R}^{\mid \mathcal{P} \mid}$, 
and the target trajectory is produced by sorting all POIs in descending order according to their scores 
$S(p \mid \mathbf{x}),~ \forall p \in \mathcal{P}$,
then picking the top $K-1$ from $\mathcal{P} \setminus s$.

First, we construct POI feature vectors $\Psi$ and labels $R$ for trajectories in training set, 
\begin{align*}
\Psi &= \left( \Psi(\mathbf{x}^{(i)}, p) \right)_{i=1,\dots,N,~p \in \mathcal{P}} = \left( \Psi_p^i \right)_{i,p}
        \in \mathbb{R}^{(N \cdot \mid \mathcal{P} \mid) \times D}, \\
   R &= \left( r(\mathbf{y}^{(i)}, p \right)_{i=1,\dots,N,~p \in \mathcal{P}} = \left(r_p^i \right)_{i,p}
        \in \mathbb{R}^{(N \cdot \mid \mathcal{P} \mid) \times 1},
\end{align*}
where $D$ is the dimension of feature vector and label $r(\mathbf{y}, p)$ is related to the location POI $p$ in trajectory $\mathbf{y}$,
\begin{equation*}
r(\mathbf{y}, p) = \sum_{j=1}^{\mid \mathbf{y} \mid} (\mid \mathcal{P} \mid - j + 1) \cdot \mathbbm{1}(y_j = p),
\end{equation*}
here $\mathbbm{1}(\cdot)$ is the indicator function, and the $j$-th POI in $\mathbf{y}$ will be scored $\mid \mathcal{P} \mid - j + 1$, 
any POI that does not appear in $\mathbf{y}$ will be scored $0$.
We can also normalise these scores, i.e., $\bar{r}(\mathbf{y}, p) = \frac{1}{\mid \mathcal{P} \mid} r(\mathbf{y}, p)$.


\subsubsection{Occurrence prediction}
\label{sec:logistic}

To begin with, we can simply ignore the order of POIs in a trajectory and just model the occurrence of a POI.
Formally, we construct binary labels
\begin{equation*}
l_p^i = \begin{cases}
+1,~p \in \mathbf{y}^{(i)}, \\
-1,~p \notin \mathbf{y}^{(i)}.
\end{cases}
\end{equation*}
and train a logistic regression model 
%\begin{flalign*} % full-length alignment (align left), note the double-& at the end of equation
%\textsc{\underline{objective}} \hspace{2em} 
\begin{equation*}
\min_{\mathbf{w}} \frac{1}{2} \mathbf{w}^\top \mathbf{w} + 
C \sum_{i=1}^N \sum_{p \in \mathcal{P}} \log \left(1 + \exp \left(- l_p^i \cdot \mathbf{w}_p^\top \Psi_p^i \right) \right), %&&
\end{equation*}
%\end{flalign*}
where $\mathbf{w}$ is an ensemble of all POI specific parameters $\mathbf{w}_p, p \in \mathcal{P}$ and $C>0$ is a regularisation constant.
We can also share parameters between POIs by assuming $\mathbf{w}_p = \mathbf{u} + \mathbf{v}_p$.

The score of POI $p$ is the probability of $p$ occurring in trajectory $\mathbf{y}$ (w.r.t. query $\mathbf{x}$)
\begin{equation*}
S(p \mid \mathbf{x})
= \mathbb{P}(p \in \mathbf{y} \mid \mathbf{x}; \mathbf{w})
= \mathbb{P}(r(\mathbf{y}, p) > 0 \mid \mathbf{x}; \mathbf{w})
= \sigma \left( \mathbf{w}_p^\top \Psi(\mathbf{x}, p) \right),
\end{equation*}
where $\sigma(z) = \frac{1}{1+\exp({-z})}$ is the logistic function.



\subsubsection{Direct rank prediction}
\label{sec:linear}

On the other hand, we can directly model the location/rank of a POI in a trajectory.
Formally, we train a linear regression model to minimise the squared loss $\|r_p^i - \hat{r}_p^i \|_2^2$, 
where the predicted rank $\hat{r}_p^i = \mathbf{w}_p^\top \Psi_p^i$,
\begin{equation*}
\min_{\mathbf{w}} \frac{1}{2} \mathbf{w}^\top \mathbf{w} + C \sum_{i=1}^N \sum_{p \in \mathcal{P}} \|r_p^i - \mathbf{w}_p^\top \Psi_p^i \|_2^2,
\end{equation*}
where parameter settings are similar to those in Section~\ref{sec:logistic}.

The score of POI $p$ is the expected rank of $p$ given query $\mathbf{x}$ 
\begin{equation*}
S(p \mid \mathbf{x})
= \mathbb{E}(\hat{r}_p \mid \mathbf{x}; \mathbf{w}) 
= \mathbf{w}_p^\top \Psi(\mathbf{x}, p).
\end{equation*}

%To learn the parameters $\mathbf{w}$, we need to solve linear equations $\Psi \cdot \mathbf{w} = R$,
%which is straightforward, i.e., $\mathbf{w} = \Psi^{-1} R$ or $\mathbf{w} = \Psi \backslash R$ by using matrix left division.

We note that when two different trajectories satisfy the same query, the feature vectors for all POIs will be the same (see Section~\ref{sec:feature})
but the labels (i.e., ranks) can be different, which may confuse this model, in this case, the prediction will be the average rank.



\subsubsection{Pairwise ranking}
\label{sec:rank}

Let $\phi(\cdot)$ be a loss function and $l_{p,p'}^i$ be a binary label
\begin{equation*}
l_{p,p'}^i = \begin{cases}
+1,~ \mathbf{y}^{(i)} ~\text{prefers}~ p, \\
-1,~ \mathbf{y}^{(i)} ~\text{prefers}~ p'.
\end{cases}
\end{equation*}
We can learn a ranking model
\begin{equation*}
\min_{\mathbf{w}} \frac{1}{2} \mathbf{w}^\top \mathbf{w} +  
C \sum_{i=1}^N \sum_{p, p' \in \mathcal{P}} \phi \left( l_{p,p'}^i \cdot \mathbf{w}^\top \left( \Psi_p^i - \Psi_{p'}^i \right) \right).
\end{equation*}

There are a number of options for the design of loss function $\phi(\cdot)$,
\begin{itemize}
\item if $\phi(z) = \max(0,~ 1-z)^2$, we are training a rankSVM with linear kernel and L2 loss, and the ranking score 
      \begin{equation*}
      S(p \mid \mathbf{x})= \mathbf{w}^\top \Psi(\mathbf{x}, p);
      \end{equation*}
\item if $\phi(z) = \log(1 + \exp(-z))$, we are training a logistic regression model, and the ranking score
      \begin{equation*}
      S(p \mid \mathbf{x})= \mathbb{P}(p \mid \mathbf{x}; \mathbf{w}) = \sigma \left(\mathbf{w}^\top \Psi(\mathbf{x}, p) \right).
      \end{equation*}
\end{itemize}

Similarly, we have a few options for the design of labels $l_{p,p'}^i$,
\begin{itemize}
\item let $c_p^i$ denotes the number of times POI $p$ was observed in trajectories satisfying query $\mathbf{x}^{(i)}$ (except the start POI), and define
      \begin{equation*}
      l_{p,p'}^i = \begin{cases}
      +1,~ c_p^i > c_{p'}^i, \\
      -1,~ c_p^i < c_{p'}^i.
      \end{cases}
      \end{equation*}
      This definition reflects whether $p$ was occurred more frequently than $p'$ for query $\mathbf{x}^{(i)}$.
\item Furthermore, we can implicitly incorporate the ranking of a POI into the binary labels,
      \begin{equation*}
      l_{p,p'}^i = \begin{cases}
      +1,~ r(\mathbf{y}^{(i)}, p) > r(\mathbf{y}^{(i)}, p'), \\
      -1,~ r(\mathbf{y}^{(i)}, p) < r(\mathbf{y}^{(i)}, p').
      \end{cases}
      \end{equation*}
\end{itemize}



\subsection{Trajectory scoring}
\label{sec:structured}

We can model the dependences between different POIs in a trajectory by employing structured prediction models,
either probabilistic models such as maximum-entropy Markov models (MEMM) and conditional random fields (CRF),
or non-probabilistic model such as structured SVM.
We model the desired trajectory with respect to query $\mathbf{x}$ as a sequence of discrete variables, 
with the first variable being observed, and each variable has $|\mathcal{P}|$ states.
To make a recommendation, we find a trajectory that achieves the highest score
\begin{equation*}
\mathbf{y}^* = \argmax_{\mathbf{y} \in \mathcal{Y}}~ f(\mathbf{x}, \mathbf{y}),
\end{equation*}
where $\mathcal{Y}$ is the set of all possible trajectory with POIs in $\mathcal{P}$ and satisfies query $\mathbf{x}$,
$f(\mathbf{x}, \mathbf{y})$ is a function that scores the compatibility between query $\mathbf{x}$ and a specific trajectory $\mathbf{y}$.



\subsubsection{Maximum-entropy Markov models}
\label{sec:memm}

For MEMM, the compatibility function $f(\mathbf{x}, \mathbf{y})$ is the probability of trajectory $\mathbf{y}$ given query $\mathbf{x}$,
\begin{equation*}
f(\mathbf{x}, \mathbf{y}) = \mathbb{P}(\mathbf{y} \mid \mathbf{x}; \mathbf{w}) 
                          = \prod_{j=2}^{\mid \mathbf{y} \mid}~
                            \frac{\exp \left(\mathbf{w}^\top \Psi_j(\mathbf{x}, y_{j-1}, y_j) \right)}
                                 {\sum_{y' \in \mathcal{P}} \exp \left(\mathbf{w}_j^\top \Psi_j(\mathbf{x}, y_{j-1}, y') \right)}.
\end{equation*}

The negative log-likelihood is 
\begin{equation*}
\ell(\mathbf{w}) = -\sum_{i=1}^N \log \mathbb{P}(\mathbf{y}^{(i)} \mid \mathbf{x}^{(i)}; \mathbf{w}) \\
                 = -\sum_{i=1}^N \sum_{j=2}^{\mid \mathbf{y}^{(i)} \mid} 
                    \mathbf{w}^\top \Psi_j(\mathbf{x}^{(i)}, y_{j-1}^{(i)}, y_j^{(i)}) +
                    \sum_{i=1}^N \sum_{j=2}^{\mid \mathbf{y}^{(i)} \mid} 
                    \log \sum_{y' \in \mathcal{P}} \exp \left(\mathbf{w}_j^\top \Psi_j(\mathbf{x}^{(i)}, y_{j-1}^{(i)}, y') \right).
\end{equation*}

To learn the parameters, we minimise the negative log-likelihood with L2 regularisation (maximum likelihood estimation (MLE))
\begin{equation*}
\min_{\mathbf{w}} \frac{1}{2} \mathbf{w}^\top \mathbf{w} + C \ell(\mathbf{w}).
\end{equation*}


\subsubsection{Conditional random fields}
\label{sec:crf}

For linear chain CRF, the compatibility function $f(\mathbf{x}, \mathbf{y})$ is also the probability of trajectory $\mathbf{y}$ given query $\mathbf{x}$,
\begin{equation*}
f(\mathbf{x}, \mathbf{y}) = \mathbb{P}(\mathbf{y} \mid \mathbf{x}; \mathbf{w}) 
= \frac{\exp \left( \mathbf{w}^\top \Psi(\mathbf{x}, \mathbf{y}) \right)}
       {\sum_{\mathbf{y}'} \exp \left( \mathbf{w}^\top \Psi(\mathbf{x}, \mathbf{y}') \right)}
= \frac{\prod_{j=2}^{\mid \mathbf{y} \mid} \exp \left( \mathbf{w}^\top \Psi_j(\mathbf{x}, y_{j-1}, y_j) \right)}
       {\sum_{\mathbf{y}'} \prod_{j=2}^{\mid \mathbf{y}' \mid} \exp \left( \mathbf{w}^\top \Psi_j(\mathbf{x}, y_{j-1}', y_j') \right)},
\end{equation*}
assuming decomposition and parameter tying
\begin{equation*}
\mathbf{w}^\top \Psi(\mathbf{x}, \mathbf{y}) = \sum_{j=2}^{\mid \mathbf{y} \mid} \mathbf{w}^\top \Psi_j(\mathbf{x}, y_{j-1}, y_j).
\end{equation*}

The negative log-likelihood is 
\begin{equation*}
\ell(\mathbf{w}) 
= -\sum_{i=1}^N \log \mathbb{P}(\mathbf{y}^{(i)} \mid \mathbf{x}^{(i)}; \mathbf{w})
= -\sum_{i=1}^N \sum_{j=2}^{\mid \mathbf{y}^{(i)} \mid} \mathbf{w}^\top \Psi_j(\mathbf{x}^{(i)}, y_{j-1}^{(i)}, y_j^{(i)}) +
   \sum_{i=1}^N \log \sum_{\mathbf{y}'} \prod_{j=2}^{\mid \mathbf{y}' \mid} \exp \left( \mathbf{w}^\top \Psi_j(\mathbf{x}^{(i)}, y_{j-1}', y_j') \right).
\end{equation*}

To learn the parameters, we minimise the negative log-likelihood with L2 regularisation (MLE)
\begin{equation*}
\min_{\mathbf{w}} \frac{1}{2} \mathbf{w}^\top \mathbf{w} + C \ell(\mathbf{w}).
\end{equation*}



\subsubsection{Structured SVM}
\label{sec:ssvm}

For structured SVM, the compatibility function $f(\mathbf{x}, \mathbf{y})$ is this linear form,
\begin{equation*}
f(\mathbf{x}, \mathbf{y}) = \mathbf{w}^\top \Psi(\mathbf{x}, \mathbf{y}),
\end{equation*}
where the $\Psi(\mathbf{x}, \mathbf{y})$ is a \emph{joint feature map} 
which captures features extracted from both query $\mathbf{x}$ and trajectory $\mathbf{y}$.

The design of joint feature $\Psi(\cdot)$ is problem specific, 
in the setting of trajectory recommendation,
assuming decomposition and parameter tying, we have
\begin{equation*}
\label{eq:jointfeature}
\mathbf{w}^\top \Psi(\mathbf{x}, \mathbf{y}) = \sum_{j=2}^{\mid \mathbf{y} \mid} 
                                               \left( \mathbf{w}_1^\top \Psi_j(\mathbf{x}, y_j) + 
                                                      \mathbf{w}_2^\top \Psi_{j-1, j}(\mathbf{x}, y_{j-1}, y_j) \right),
\end{equation*}
where $\mathbf{w}_1$ and $\mathbf{w}_2$ are parameter vectors,
$\Psi_j$ is a feature vector of POI $y_j$ with respect to query $\mathbf{x}$ (Table~\ref{tab:poifeature}),
$\Psi_{j-1,j}$ is a pairwise feature vector that captures the affinity of transition from POI $y_{j-1}$ to POI $y_j$ and
here we use the transition probabilities between individual POI properties as described in Table~\ref{tab:tranfeature}.
This joint feature design shares parameters among POIs/transitions in a trajectory.

To learn the parameters, we train the structured SVM by optimising a quadratic program (QP),
\begin{equation}
\label{eq:nslackform}
\begin{aligned}
\min_{\mathbf{w}, ~\bm{\xi} \ge 0} ~& \frac{1}{2} \mathbf{w}^\top \mathbf{w} + \frac{C}{n} \sum_{i=1}^n \xi_i \\
s.t.~~ ~& \mathbf{w}^\top \Psi(\mathbf{x}^{(i)}, \mathbf{y}^{(i)}) - \mathbf{w}^\top \Psi(\mathbf{x}^{(i)}, \bar{\mathbf{y}}) \ge 
       \Delta(\mathbf{y}^{(i)}, \bar{\mathbf{y}}) - \xi_i, ~\forall i
\end{aligned}
\end{equation}
where $\mathbf{w} = [\mathbf{w}_1, \mathbf{w}_2]^\top$ is the parameter vector, $C > 0$ is a regularisation constant, 
and $\Delta(\mathbf{y}, \bar{\mathbf{y}})$ is a discrepancy function that measures the loss 
for prediction $\bar{\mathbf{y}}$ given ground truth $\mathbf{y}$, and $\xi_i$
is a slack variable that represents the \emph{hinge loss} associated with the prediction for the $i$-th example~\cite{tsochantaridis2005large},
\begin{equation*}
\label{eq:nslackloss}
\xi_i = \max \left( 0,~ 
        \max_{\bar{\mathbf{y}} \in \mathcal{Y}} 
        \left\{ \Delta(\mathbf{y}_i, \bar{\mathbf{y}}) + \mathbf{w}^\top \Psi(\mathbf{x}^{(i)}, \bar{\mathbf{y}}) \right\} -
        \mathbf{w}^\top \Psi(\mathbf{x}^{(i)}, \mathbf{y}^{(i)}) \right).
\end{equation*}
%This formulation is called "$n$-slack" as we have one slack variable for each example in training set.

We can rewrite the constraint of problem (\ref{eq:nslackform}) as
\begin{equation*}
\mathbf{w}^\top \Psi(\mathbf{x}^{(i)}, \mathbf{y}^{(i)}) + \xi_i \ge
          \max_{\bar{\mathbf{y}} \in \mathcal{Y}} 
          \left\{\mathbf{w}^\top \Psi(\mathbf{x}^{(i)}, \bar{\mathbf{y}}) + \Delta(\mathbf{y}^{(i)}, \bar{\mathbf{y}}) \right\},~ \forall i.
\end{equation*}
where the right hand side is the \emph{loss-augmented inference}.

To solve problem (\ref{eq:nslackform}), one option is simply enumerating all constraints, feeding the problem into a standard QP solver.
However, this approach is impractical as there is a constraint for every possible label $\bar{\mathbf{y}}$.
Instead, a cutting-plane algorithm is employed which repeatedly solves QP (\ref{eq:nslackform}) with respect to different set of constraints, 
and each iteration solves the loss-augmented inference and generates a new constraint that helps shrink the feasible region of the problem, 
until a specified precision $\varepsilon$ is achieved~\cite{joachims2009predicting}.

The loss-augmented inference for trajectory recommendation is equivalent to 
find a maximum weighted loop-less path with exactly $k$ edges in a complete weighted (both nodes and edges) graph, which is NP-hard (need proof).
To solve the loss-augmented inference, we can formulated it as an integer linear program (ILP) and solve it using a ILP solver,
or use lazy constraint generation/cutting plane technique with an LP solver.
Moreover, we can use list Viterbi algorithm~\cite{nill1995list} or 
employ heuristics such as the Christofides algorithm~\cite{christofides1976} when the problem has the triangle inequality property 
(which is indeed for trajectories).


\eat{
\subsection{Other models}
\label{sec:other}
Label ranking model,
Plackett-Luce probabilistic ranking


\section{Evaluation metrics}
\label{sec:evaluation}
F1 score on points
F1 score on pairs
Kendall's tau, all POIs not appeared in trajectory are ranked last (and share the same rank).
}


\begin{table*}[ht]
\caption{Features of POI $p$ with respect to query $(s,k)$}
\label{tab:poifeature}
\centering
\setlength{\tabcolsep}{10pt} % tweak the space between columns
\begin{tabular}{l|l} \hline
\textbf{Feature}  & \textbf{Description} \\ \hline
\texttt{category}               & one-hot encoding of the category of $p$ \\
\texttt{neighbourhood}          & one-hot encoding of the POI cluster that $p$ resides in \\
\texttt{popularity}             & logarithm of POI popularity of $p$ \\
\texttt{nVisit}                 & logarithm of the total number of visit by all users at $p$ \\
\texttt{avgDuration}            & logarithm of the average duration at $p$ \\ \hline
\texttt{trajLen}                & trajectory length $k$, i.e., the number of POIs required \\
\texttt{sameCatStart}           & $1$ if the category of $p$ is the same as that of $s$, $-1$ otherwise \\
\texttt{sameNeighbourhoodStart} & $1$ if $p$ resides in the same POI cluster as $s$, $-1$ otherwise \\
\texttt{distStart}              & distance between $p$ and $s$, calculated using the Haversine formula \\
\texttt{diffPopStart}           & real-valued difference in POI popularity of $p$ from that of $s$ \\
\texttt{diffNVisitStart}        & real-valued difference in the total number of visit at $p$ from that at $s$ \\
\texttt{diffDurationStart}      & real-valued difference in average duration at $p$ from that at $s$ \\
\hline
\end{tabular}
\end{table*}


\section{Features}
\label{sec:feature}

The POI and query specific features we extracted from trajectories are shown in Table~\ref{tab:poifeature},
features that describe the transition preference between different POIs are shown in Table~\ref{tab:tranfeature}.


\begin{table}[ht]
\caption{POI features used to estimate the (feature-wise) transition probabilities}
\label{tab:tranfeature}
\centering
%\setlength{\tabcolsep}{28pt} % tweak the space between columns
\begin{tabular}{l|l} \hline
\textbf{Feature}       & \textbf{Description} \\ \hline
\texttt{category}      & category of POI \\
\texttt{neighbourhood} & the cluster that a POI resides in \\
\texttt{popularity}    & (discretised) popularity of POI \\
\texttt{nVisit}        & (discretised) total number of visit at POI \\
\texttt{avgDuration}   & (discretised) average duration at POI \\ \hline
\end{tabular}
\end{table}



\bibliographystyle{ieeetr}
\bibliography{ref}

\end{document}
