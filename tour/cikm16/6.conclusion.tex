\section{Conclusion}
\label{sec:conclusion}
\secmoveup

In this paper, we propose an approach to recommend trajectories
by jointly optimising points and routes.
This is in contrast to related work which looks at POI %recommendation 
or next location recommendation.
Point preferences are learned by ranking according to POI and query features,
and transition probabilities between POIs are learned from previous trajectories extracted
from social media.
We investigate the weaknesses of a naive maximum likelihood sequence approach (which
may recommend sub-tours) and propose an improved sequence prediction method,
our feature driven approach allows a natural combination of POI ranks and routes.
%and we proposed two ways (directly combining them, or learning a structured SVM)
%and we propose a probabilistic model for unifying the two models.
We argue that one should measure performance also with respect to the visiting order of POIs, 
and suggest a new pairs-F$_1$ metric.
We empirically evaluated our tour recommendation approaches on five datasets extracted from
Flickr photos, and demonstrate that our method improves on prior work, 
on both the traditional F$_1$ metric and our proposed performance measure.
Our promising results from learning points and routes for trajectory recommendation suggests
that research in this domain should consider both information sources simultaneously.
%Since our machine learning method captures domain knowledge in terms of the features,
%it does not rely on the fact that the current task uses spatial and social information,
%therefore it may be more widely applicable to other trajectory recommendation tasks.


\begin{figure*}[t]
	\centering
	\includegraphics[width=\textwidth]{fig/example-tour.pdf}
	\caption{Different recommendations from algorithm variants.
    See the main text in Section~\ref{sec:example} for description.}
	\label{fig:exampleresult}
	%\captionmoveup
\end{figure*}
