%!TEX root = main.tex

\section{Discussion and Conclusion}
\label{sec:conclusion}
%\secmoveup

In this paper, we propose an approach to recommend trajectories
by jointly optimising point preferences and routes.
This is in contrast to related work which looks at only POI %recommendation
or next location recommendation.
Point preferences are learned by ranking according to POI and query features,
and factorised transition probabilities between POIs
are learned from previous trajectories extracted
from social media.
We investigate the maximum likelihood sequence approach (which
may recommend sub-tours) and propose an improved sequence recommendation method.
Our feature driven approach naturally allows learning the combination of POI ranks and routes.
%and future work includes using recent advances in structured prediction to jointly learn the combined model.
%Furthermore, one could investigate the setting
%where instead of recommending trajectories
%with a certain number of POIs, we recommend trajectories that satisfy a certain time limit.

We argue that one should measure performance with respect to the visiting order of POIs,
and suggest a new pairs-F$_1$ metric.
We empirically evaluate our tour recommendation approaches on five datasets extracted from
Flickr photos, and demonstrate that our method improves on prior work,
in terms of both the traditional F$_1$ metric and our proposed performance measure.
Our promising results from learning points and routes for trajectory recommendation suggests
that research in this domain should consider both information sources simultaneously.
%Since our machine learning method captures domain knowledge in terms of the features,
%it does not rely on the fact that the current task uses spatial and social information,
%therefore it may be more widely applicable to other trajectory recommendation tasks.


\begin{figure*}[t]
	\centering
	\includegraphics[width=\textwidth]{fig/example-tour.pdf}
	\caption{Different recommendations from algorithm variants.
    See the main text in Section~\ref{sec:example} for description.}
	\label{fig:exampleresult}
	%\captionmoveup
\end{figure*}
